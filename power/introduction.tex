\section{Introduction and Planning}

For Tiberius III a complete redesign of the power management was required. The amount of new hardware, new motors, new batteries and different wiring meant that everything had to be redone.  The plans set out in the preliminary design review were the first point of reference for the redesign but inevitably various changes were made as the project progressed.
Some of these changes included a circular LED array to display use of certain hardware, a new set of relays and fuses to control the power and a reconfiguring of the wiring and layout of units to allow the design to be much more modular and robust.


\subsection{Tiberius II}

The previous model of Tiberius had some basic power monitoring functions which included relays to turn on and off parts of the vehicle and a seven segment display to show the charge currently stored in the batteries.

\subsection{Aims and Objectives}
The aims and objectives for Tiberius III were to have some robust, controllable power management, easy to use and clearly visible power monitoring and a high level of safety and fail safe mechanisms on board the vehicle.  
This meant adapting some features, redesigning others and adopting completely new ones.  Our primary aim was to adopt new lighter weight, higher capacity LiFe batteries.  This was to reduce weight and improve battery life.  Secondary aims were to update the database to contain information about battery charge  and power consumption.  Finally, further aims were to improve the efficiency of certain subsystems and use more energy efficient hardware wherever possible.
