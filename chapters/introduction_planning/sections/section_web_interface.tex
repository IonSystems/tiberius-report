% Introduction and Planning
% Web Interface
% Author: Cameron Craig
% Date: 03/05/2016

\section{Web Interface}
\pagestyle{cameron}

\subsection{Introduction}
The web interface is designed to be the main communication mechanism between the user and Tiberius. In order for the web interface to be successful, other components such as the Control API, Navigation Algorithms, Communications network, and in-memory database (web interface dependencies) must also be successfully implemented. The high reliance on other components meant that development time was shared among the integral components, ensure a compatible platform for the web interface by the end of the project.

The web interface was built using the originally specialized web framework - Django. The primary reason to use Django was because it is written in Python. It was initially decided to make development of Tiberius as friendly as possible, and writing as much code as possible in the same language - without compromising on functionality.

User guides and tutorials have been made available online at \url{https://tiberius-robot.readthedocs.io/en/latest/}. These are designed to aid future developers coming to grips with the web interface design. A summary of the online documentation is included in Appendix \ref{app:online-docs}.

\subsection{Planning}
The development of the web interface was planned from the outset of the project. Progress of the development was managed on Github, using issue tracking. A full description of the initial plans and specification can be found in last semester's preliminary design review \cite[p. ~132]{tibby-lit-review}.



