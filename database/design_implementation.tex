\section{Design and Implementation}

\subsection{Table definitions}
Each table has unique set of columns to suit the sensor data.
Every table has 2 columns in common. One is the id, this column represents the reading and provides a way to identify each reading. The first reading has id 1, and it increments with each reading. The other column in the timestamp, this is the time when the sensor data was inserted into the database - which is very close to the time the sensor data was aquired. Therefore the timestamp can be used to determine how old a sensor reading is.
\newline
Each table is defined in its own class.
Below is an example of the GPS table. The reason for the column names is expanded later in the GPS section.
\begin{lstlisting}[style=custompython]
class GPSTable(Table):
    table_name = "gps_reading"
    columns = {
        'id': 'int primary key',
        'latitude': 'float',
        'longitude': 'float',
        'altitude': 'float',
        'gps_qual': 'int',
        'num_sats': 'int',
        'dilution_of_precision': 'float',
        'velocity': 'float',
        'fixmode': 'int',
        'timestamp': 'float'}
\end{lstlisting}


\subsubsection{Ultrasonics}

\begin{table}[!htb]
\centering
\begin{tabular}{|l|l|l|l|l|l|l|l|}
\hline
id              & fl    & fc    & fr    & rl    & rc    & rr    & timestamp \\ \hline
int primary key & float & float & float & float & float & float & float \\ \hline
\end{tabular}
\caption{IMDB: Ultrasonics Table}
\label{tab:db-ultrasonics}
\end{table}


\begin{itemize}
\item{\textbf{fl}}: distance in millimeters from front left sensor
\item{\textbf{fc}}: distance in millimeters from  front centre sensor
\item{\textbf{fr}}: distance in millimeters from  front right sensor
\item{\textbf{rl}}: distance in millimeters from rear left sensor
\item{\textbf{rc}}: distance in millimeters from rear centre sensor
\item{\textbf{rr}}: distance in millimeters from rear right sensor
\end{itemize}

\subsubsection{LIDAR}
\iffalse
\gls{LIDAR} data read using SDK (C++)
Changed the output to return a JSON format
Python dictionary in this format...
Small error with the last comma breaking the JSON object so used python \lstinline{data.replace(',\n' , ']')} which replaced the last comma with a ]

Lidar class validates the data
 - removes data (angle) which is obscured by the antenna
\fi



% Please add the following required packages to your document preamble:
% \usepackage{booktabs}
\begin{table}[!htb]
\centering
\begin{tabular}{|l|l|l|l|l|l|l|}
\hline
id              & start\_flag & angle & distance & quality & reading\_iteration & timestamp \\ \hline
int primary key & varchar     & float & float    & float   & int            & float\\ \hline
\end{tabular}
\caption{IMDB: LIDAR Table}
\label{tab:db-lidar}
\end{table}
\begin{itemize}
\item{\textbf{start\_flag}}: a flag which is toggled on the last reading iteration of a full spin. It is included as it is produces by the SDK, however currently it is not used.
\item{\textbf{angle}}: the angle which the reading was taken at.
\item{\textbf{distance}}: the distance of an object that is line the lidar's line of sight.
\item{\textbf{qualitty}}: the measure of certanity that there is an object is that location
\item{\textbf{reading\_iteration}} increments after each full spin. This means that when querying the database, the user can ensure that all the readings come from the same 360 degree spin.
\end{itemize}


\subsubsection{GPS}

% Please add the following required packages to your document preamble:
% \usepackage{booktabs}
% Please add the following required packages to your document preamble:
% \usepackage{booktabs}
\begin{table}[!htb]
\centering
\begin{tabular}{|l|l|l|l|l|l|l|l|l|}
\hline
id              & latitude & longitude & altitude & gps\_qual & num\_sats & velocity & fixmode & timestamp \\ \hline
int primary key & float    & float     & float    & int       & int       & float    & int     & float\\ \hline
\end{tabular}
\caption{IMDB: GPS Table}
\label{tab:db-gps}
\end{table}
\begin{itemize}
\item{\textbf{latitude}}: the latitude specifies a location's distance north or south of the equator.
\item{\textbf{longitude}}: the longitude specifies the location's distance east or west of the Prime Meridian.
\item{\textbf{altitude}}: the height of the robotin relation to sea level or ground level.
\item{\textbf{gps\_qual}}: the value of the signal
\item{\textbf{num\_stats}}: the number of satellites the gps unit has communication with
\item{\textbf{velocity}}: the velocity which the robot is travelling at
\item{\textbf{fixmode}}: boolean value which represents whether the gps has a location or not.  
\end{itemize}


\subsubsection{Compass}
% Please add the following required packages to your document preamble:
% \usepackage{booktabs}
\begin{table}[!htb]
\centering
\begin{tabular}{|l|l|l|l|l|l|l|}
\hline
id              & heading & tilt  & pitch & roll  & temperature & timestamp \\ \hline
int primary key & float   & float & float & float & float       & float \\ \hline
\end{tabular}
\caption{IMDB: Compass Table}
\label{tab:db-compass}
\end{table}
\begin{itemize}
\item{\textbf{heading}}: the magnetic bearing that the front of the robot is pointing to
\item{\textbf{tilt}}: rotation around the vertical axis
\item{\textbf{pitch}}: rotation around the side to side axis
\item{\textbf{roll}}: rotation around the front to back axis
\item{\textbf{temperature}}: temperature reading of the compass
\end{itemize}


\subsubsection{Motors}
% Please add the following required packages to your document preamble:
% \usepackage{booktabs}
\begin{table}[!htb]
\centering
\caption{Motors table}
\label{tab:db-motors}
\begin{tabular}{|l|l|l|l|l|l|}
\toprule
id              & front\_left & front\_right & rear\_left & rear\_right & timestamp \\ \midrule
int primary key & float       & float        & float      & float       & float
\end{tabular}
\caption{IMDB: Motors table}
\label{tab:db-motors}
\end{table}
\begin{itemize}
\item{\textbf{front\_left}}: the speed the front left motor is set at
\item{\textbf{front\_right}}: the speed the front right motor is set at
\item{\textbf{rear\_left}}: the speed the rear left motor is set at
\item{\textbf{rear\_right}}: the speed the rear right motor is set at
\end{itemize}


\subsubsection{Steering}
% Please add the following required packages to your document preamble:
% \usepackage{booktabs}
\begin{table}[!htb]
\centering
\caption{Steering table}
\label{tab:db-steering}
\begin{tabular}{|l|l|l|l|l|l|}
\toprule
id              & front\_left & front\_right & rear\_left & rear\_right & timestamp \\ \midrule
int primary key & float       & float        & float      & float       & float
\end{tabular}
\caption{IMDB: Steering Table}
\label{tab:db-steering}
\end{table}
\begin{itemize}
\item{\textbf{front\_left}}: the angle the front left wheel is set at
\item{\textbf{front\_right}}: the angle the front right wheel is set at
\item{\textbf{rear\_left}}:  the angle the rear left wheel is set at
\item{\textbf{rear\_right}}:  the angle the rear right wheel is set at
\end{itemize}

\subsubsection{Arm}
% Please add the following required packages to your document preamble:
% \usepackage{booktabs}
\begin{table}[!htb]
\centering
\begin{tabular}{|l|l|l|l|l|l|l|l|}
\toprule
id              & X     & Y     & Z     & waist & elbow & shoulder & timestamp \\ \midrule
int primary key & float & float & float & float & float & float    & float
\end{tabular}
\caption{IMDB: Arm Table}
\label{tab:db-arm}
\end{table}
\begin{itemize}
\item{\textbf{X}}: the x position of the gripper 
\item{\textbf{Y}}: the y position of the gripper
\item{\textbf{Z}}: the z position of the gripper
\item{\textbf{waist}}: the angle of the waist
\item{\textbf{elbow}}: the angle of the elbow
\item{\textbf{shoulder}}: the angle of the shoulder
\end{itemize}

\subsection{Creating database}
Several functions are called when creating the tables. On normal usage the table are created when start\_tiberius.py is run. In start Tiberius all the configured files are called




It is then created with unique columns described in the next subsections.

\begin{lstlisting}[style=custompython]
def drop_create(poly, table):
    '''
    Drop a table then create it, this ensures the table has the latest column
    definitions and is initiall free of any data.
    '''
    try:
        poly.drop(table.table_name)
    except poly.NoSuchTableError:
        # Table didn't exist previously, no worries!
        pass
    try:
        poly.create(
            table.table_name,
            table.columns
        )
    except poly.OperationalError:
        print "something went wrong... "
    except poly.TableAlreadyExistsError:
        print table.table_name + " already exists."
\end{lstlisting}

As can be seen the table is dropped before being created. This means is Tiberius was previously running and Tiberius had not been switched off

\subsection{Insert}
Insert is used to add a new row to of data to into a table. The data is packed into one variable using python dictionary. The data is then extracted by specifying the keyword, which the data is related to.  

\begin{lstlisting}[style=custompython]
def insert_gps_reading(poly, id, data):
    poly.insert(
        GPSTable.table_name,
        {
            'id': id,
            'latitude': data['latitude'],
            'longitude': data['longitude'],
            'gps_qual': data['gps_qual'],
            'num_sats': data['num_sats'],
            'dilution_of_precision': data['dilution_of_precision'],
            'velocity': data['velocity'],
            'fixmode': data['fixmode'],
            'timestamp': data['timestamp']
        }
    )
\end{lstlisting}

\subsection{Update}
Update is used whe only one value in a certain row needs changed, and all other rows remain the same. This is not currently used, as the need to alter a reading has never arised. 

\subsection{Overwrite}
Overwrite allow a compelete row to be overwritten with new information. The id remains the same, as the function calling overwrite has to use the id inorder to specify which row it wants overwrite.

\begin{lstlisting}[style=custompython]
def overwrite_gps_reading(poly, id, data):
    poly.update(
        GPSTable.table_name,
        {
            # 'id': id,
            'latitude': data['latitude'],
            'longitude': data['longitude'],
            'gps_qual': data['gps_qual'],
            'num_sats': data['num_sats'],
            'dilution_of_precision': data['dilution_of_precision'],
            'velocity': data['velocity'],
            'fixmode': data['fixmode'],
            'timestamp': time.time()
        },
        {
            'clause': 'WHERE',
            'data': [
                {
                    'column': 'id',
                    'assertion': '=',
                    'value': id
                }
            ]
        }
    )
\end{lstlisting}

\subsection{Polyhedra wrapper}





\subsection{Procedure}
\subsubsection{Starting database}
Upon start up the database will not be running. It is therefore necessary to run the start tiberius script to initiate the database and start the threads. The database is started using the command shown below.

\begin{lstlisting}[style=custompython]
database = Popen("/home/pi/poly9.0/linux/raspi/bin/rtrdb -r data_service=8001 db", shell=True, stdout=PIPE)
\end{lstlisting}

The tables for all the sensors are then created. This is done by calling the create function described above. Next the threads are started to continuously update the database. Threads are only started if the sensor is installed - this is ensured by the configuration file. For more detail on the configuration file see configuration  (\ref{configuration}). The motor and arm are actuators rather than sensors, and therefore are employ a different technique to update the value; this technique is further described in the Decorators section (\ref{decoratordefinition})

\subsubsection{Sensor threads}
Each sensor has its own thread; the purpose of which is to collect the data from the sensor it is assigned to. The threads run concurrently and have a their own time delay to reduce the workload.
Below is a shortened version of the GPS thread.
\begin{lstlisting}[style=custompython]
while True:
  gps_data = gps.read_gps()
  if gps_data is not False:
    if gps_read_id < GPS_NUMBER_OF_READINGS:
   		ins.insert_gps_reading(self.poly, gps_read_id, gps_data)
    else:  # start updating results
   		gps_update_id = gps_read_id % GPS_NUMBER_OF_READINGS
    	up.overwrite_gps_reading(self.poly, gps_update_id, gps_data)
    gps_read_id += 1
    no_data_time = 0
\end{lstlisting}

old data in the database is overwritten with new data TODO:fill this in

\subsubsection{Actuator decorators}
Unlike the sensors, the actuator database should not be updated in a periodic manner. This is because the actuators don't change unless specified by the control software. Therefore decorators are placed on the functions that alter an actuators position.
\newline
Decorators dynamically alter the functionality of a function, method or class without having to directly use subclasses \cite{decoratordefinition}. A decorator function is called, whenever the function it is decorating is called. So for this implementation, whenever the function to move an actuator is called, the function to update the actuators location is called.
\newline
The three function to move the arm are:
\begin{lstlisting}[style=custompython]
def rotate_waist(self,change, angle=None):  
\end{lstlisting}
\begin{lstlisting}[style=custompython]
def move_shoulder(self, change, angle=None):  
\end{lstlisting}
\begin{lstlisting}[style=custompython]
def move_elbow(self, change, angle=None):  
\end{lstlisting}
Therefore each one of these function is decorated with the database arm update function. This function takes the new values of X, Y, Z, waist, elbow, shoulder and writes it to the database. 
\newline







