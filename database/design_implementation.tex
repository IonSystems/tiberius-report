\section{Design and Implementation}

\subsection{Table definitions}
Each table has unique set of columns to suit the transducer data; this is required because different transducers produce different data which requires it's unique format. Two columns in every set of columns is the same across all tables, these are the id and the timestamp. 
\newline
The id is a unique number assigned to each row; it is necessary as it allows a mechanism to specify each row. The first insert into the table has id one, and thereafter each new insert has the id incremented from the previous id. Therefore the id for the second entry would be two.
\newline
The timestamp associates a time with every reading. The timestamp is taken from when the data is inserted into the database; since the time taken between acquiring data and inserting it is negligible, it can be assumed that the timestamp correlates to the time the transducer data was read. Therefore the timestamp can be used to determine how old a transducer reading is, which is necessary especially with the sensor data, as the environment may have changed significantly after the last reading.
\newline
To increase modularity and readability, each table is defined as its own class. Below is an example of the GPS table. 
\begin{lstlisting}[style=custompython]
class GPSTable(Table):
    table_name = "gps_reading"
    columns = {
        'id': 'int primary key',
        'latitude': 'float',
        'longitude': 'float',
        'altitude': 'float',
        'gps_qual': 'int',
        'num_sats': 'int',
        'dilution_of_precision': 'float',
        'velocity': 'float',
        'fixmode': 'int',
        'timestamp': 'float'}
\end{lstlisting}

The database structure remains compatible with SQL, however the examples provided all use polyhedra database as it is the selected choice.

\subsubsection{Ultrasonics}
\begin{table}[!htb]
\centering
\begin{tabular}{|l|l|l|l|l|l|l|l|}
\hline
id              & fl    & fc    & fr    & rl    & rc    & rr    & timestamp \\ \hline
int primary key & float & float & float & float & float & float & float \\ \hline
\end{tabular}
\caption{IMDB: Ultrasonics Table}
\label{tab:db-ultrasonics}
\end{table}


\begin{itemize}
\item{\textbf{fl}}: distance in millimetres from front left sensor
\item{\textbf{fc}}: distance in millimetres from  front centre sensor
\item{\textbf{fr}}: distance in millimetres from  front right sensor
\item{\textbf{rl}}: distance in millimetres from rear left sensor
\item{\textbf{rc}}: distance in millimetres from rear centre sensor
\item{\textbf{rr}}: distance in millimetres from rear right sensor
\end{itemize}

\subsubsection{LIDAR}
\begin{table}[!htb]
\centering
\begin{tabular}{|l|l|l|l|l|l|l|}
\hline
id              & start\_flag & angle & distance & quality & reading\_iteration & timestamp \\ \hline
int primary key & varchar     & float & float    & float   & int            & float\\ \hline
\end{tabular}
\caption{IMDB: LIDAR Table}
\label{tab:db-lidar}
\end{table}
\begin{itemize}
\item{\textbf{start\_flag}}: a flag which is toggled on the last reading iteration of a full spin. It is included as it is produces by the SDK, however currently it is not used.
\item{\textbf{angle}}: the angle which the reading was taken at.
\item{\textbf{distance}}: the distance of an object that is line the lidar's line of sight.
\item{\textbf{qualitty}}: the measure of certanity that there is an object is that location
\item{\textbf{reading\_iteration}} increments after each full spin. This means that when querying the database, the user can ensure that all the readings come from the same 360 degree spin.
\end{itemize}


\subsubsection{GPS}
\begin{table}[!htb]
\centering
\begin{tabular}{|l|l|l|l|l|l|l|l|l|}
\hline
id              & latitude & longitude & altitude & gps\_qual & num\_sats & velocity & fixmode & timestamp \\ \hline
int primary key & float    & float     & float    & int       & int       & float    & int     & float\\ \hline
\end{tabular}
\caption{IMDB: GPS Table}
\label{tab:db-gps}
\end{table}
\begin{itemize}
\item{\textbf{latitude}}: the latitude specifies a location's distance north or south of the equator.
\item{\textbf{longitude}}: the longitude specifies the location's distance east or west of the Prime Meridian.
\item{\textbf{altitude}}: the height of the robotin relation to sea level or ground level.
\item{\textbf{gps\_qual}}: the value of the signal
\item{\textbf{num\_stats}}: the number of satellites the gps unit has communication with
\item{\textbf{velocity}}: the velocity which the robot is travelling at
\item{\textbf{fixmode}}: boolean value which represents whether the gps has a location or not.  
\end{itemize}


\subsubsection{Compass}
% Please add the following required packages to your document preamble:
% \usepackage{booktabs}
\begin{table}[!htb]
\centering
\begin{tabular}{|l|l|l|l|l|l|l|}
\hline
id              & heading & tilt  & pitch & roll  & temperature & timestamp \\ \hline
int primary key & float   & float & float & float & float       & float \\ \hline
\end{tabular}
\caption{IMDB: Compass Table}
\label{tab:db-compass}
\end{table}
\begin{itemize}
\item{\textbf{heading}}: the magnetic bearing that the front of the robot is pointing to
\item{\textbf{tilt}}: rotation around the vertical axis
\item{\textbf{pitch}}: rotation around the side to side axis
\item{\textbf{roll}}: rotation around the front to back axis
\item{\textbf{temperature}}: temperature reading of the compass
\end{itemize}


\subsubsection{Motors}
% Please add the following required packages to your document preamble:
% \usepackage{booktabs}
\begin{table}[!htb]
\centering
\begin{tabular}{|l|l|l|l|l|l|}
\hline
id              & front\_left & front\_right & rear\_left & rear\_right & timestamp \\ \hline
int primary key & float       & float        & float      & float       & float \\ \hline
\end{tabular}
\caption{IMDB: Motors table}
\label{tab:db-motors}
\end{table}
\begin{itemize}
\item{\textbf{front\_left}}: the speed the front left motor is set at
\item{\textbf{front\_right}}: the speed the front right motor is set at
\item{\textbf{rear\_left}}: the speed the rear left motor is set at
\item{\textbf{rear\_right}}: the speed the rear right motor is set at
\end{itemize}


\subsubsection{Steering}
% Please add the following required packages to your document preamble:
% \usepackage{booktabs}
\begin{table}[!htb]
\centering
\begin{tabular}{|l|l|l|l|l|l|}
\hline
id              & front\_left & front\_right & rear\_left & rear\_right & timestamp \\ \hline
int primary key & float       & float        & float      & float       & float \\ \hline
\end{tabular}
\caption{IMDB: Steering Table}
\label{tab:db-steering}
\end{table}
\begin{itemize}
\item{\textbf{front\_left}}: the angle the front left wheel is set at
\item{\textbf{front\_right}}: the angle the front right wheel is set at
\item{\textbf{rear\_left}}:  the angle the rear left wheel is set at
\item{\textbf{rear\_right}}:  the angle the rear right wheel is set at
\end{itemize}
%******************ARM*****************************
\subsubsection{Arm}

\begin{table}[!htb]
\centering
\begin{tabular}{|l|l|l|l|l|l|l|l|}
\hline
id              & X     & Y     & Z     & waist & elbow & shoulder & timestamp \\ \hline
int primary key & float & float & float & float & float & float    & float \\ \hline
\end{tabular}
\caption{IMDB: Arm Table}
\label{tab:db-arm}
\end{table}

\begin{itemize}
\item{\textbf{X}}: the x position of the gripper 
\item{\textbf{Y}}: the y position of the gripper
\item{\textbf{Z}}: the z position of the gripper
\item{\textbf{waist}}: the angle of the waist
\item{\textbf{elbow}}: the angle of the elbow
\item{\textbf{shoulder}}: the angle of the shoulder
\end{itemize}

\subsubsection{Battery}

\begin{table}[!htb]
\centering
\begin{tabular}{|l|l|l|l|l|l|l|l|}
\hline
id              & monitor     & volts     & amps     & power & time  & amp\_hours  & timestamp \\ \hline
int primary key & float 	  & float 	  & float    & float & float & float       & float \\ \hline
\end{tabular}
\caption{IMDB: Battery Table}
\label{tab:db-arm}
\end{table}

\begin{itemize}
\item{\textbf{monitor}}: selected battery being monitored
\item{\textbf{volts}}: the output voltage of the battery
\item{\textbf{amps}}: the output current of the battery
\item{\textbf{power}}: the power being drawn
\item{\textbf{time}}: the estimation of time left 
\item{\textbf{amp\_hours}}: the amount of energy charge in a battery that will allow one ampere of current to flow for one hour.
\end{itemize}


%***********************Creating*****************
\subsection{Creating table}
\label{create}
Creating a database refers to the act generating a new table. This act has the possibility of breaking if a table with the same name already exists. With Tiberius this case may arise, if the user has restarted the Tiberius software without resetting the raspberry pi - thus retaining the old tables. Since the old tables do not contain any pertinent information they can be deleted with out regard. Therefore before creating a new table, it is checked to see if it already exists, if it does it is deleted and the new table is created. 

\begin{lstlisting}[style=custompython]
def drop_create(poly, table):
    '''
    Drop a table then create it, this ensures the table has the latest column
    definitions and is initiall free of any data.
    '''
    try:
        poly.drop(table.table_name)
    except poly.NoSuchTableError:
        # Table didn't exist previously, no worries!
        pass
    try:
        poly.create(
            table.table_name,
            table.columns
        )
    except poly.OperationalError:
        print "something went wrong... "
    except poly.TableAlreadyExistsError:
        print table.table_name + " already exists."
\end{lstlisting}


\subsection{Insert}
Insert is the action of adding a new row to an existing table. 
To do this the data is packed into one variable in the format of a python dictionary. Dictionaries are indexed by keys, which can be any immutable type, in the Tiberius software only strings are used. Dictionaries can be thought of as an unordered set of key: value pairs, with the requirement that the keys are unique. The data can then be extracted when given the corresponding key.

\begin{lstlisting}[style=custompython]
def insert_gps_reading(poly, id, data):
    poly.insert(
        GPSTable.table_name,
        {
            'id': id,
            'latitude': data['latitude'],
            'longitude': data['longitude'],
            'gps_qual': data['gps_qual'],
            'num_sats': data['num_sats'],
            'dilution_of_precision': data['dilution_of_precision'],
            'velocity': data['velocity'],
            'fixmode': data['fixmode'],
            'timestamp': data['timestamp']
        }
    )
\end{lstlisting}

\subsection{Update}
The update function is used when only a certain value in a row needs modified. This function modifies that certain value with the new value and leaves all other values in the row (and table) unchanged. This function is not used with the transducer data itself, since there is no desire to modify old readings, however it is used for the validity table which is consistently being updated. 


\subsection{Overwrite}
Overwrite allows a full row of data to be overwritten with new data. The only row which remains constant is the id. The id remains the same since the function which calls overwrite needs to know which id to specify. 

\begin{lstlisting}[style=custompython]
def overwrite_gps_reading(poly, id, data):
    poly.update(
        GPSTable.table_name,
        {
            # 'id': id,
            'latitude': data['latitude'],
            'longitude': data['longitude'],
            'gps_qual': data['gps_qual'],
            'num_sats': data['num_sats'],
            'dilution_of_precision': data['dilution_of_precision'],
            'velocity': data['velocity'],
            'fixmode': data['fixmode'],
            'timestamp': time.time()
        },
        {
            'clause': 'WHERE',
            'data': [
                {
                    'column': 'id',
                    'assertion': '=',
                    'value': id
                }
            ]
        }
    )
\end{lstlisting}

\subsection{Polyhedra wrapper}

TODO: all of this shit





\subsection{Procedure}
\label{database_process}
The procedure the Tiberius software follows in order to run the database can be thought of as four distinct sections - starting the database, creating the table, running the sensor threads and the decorators for the actuator functions. 
\newline
The start Tiberius script does the following
\begin{itemize}
\item run the database
\item create the tables
\item start the sensor threads
\end{itemize} 
Therefore all the database set up is achieved by running the script since the actuator decorators do not require any set up.

\subsubsection{Starting database}
When Tiberius is first powered up, and consequently the raspberry pi is first powered  up, the database will not be running. 
It is therefore necessary in the script to initiate the database; this is done by the following command.

\begin{lstlisting}[style=custompython]
database = Popen("/home/pi/poly9.0/linux/raspi/bin/rtrdb -r data_service=8001 db", shell=True, stdout=PIPE)
\end{lstlisting}

\subsubsection{Creating tables}
The table for every transducer must be created at the start, regardless if the transducer is installed or not. This is done to ensure there are not errors further down the line, by 

The tables for all the sensors is created using the create function described above in section \ref{create}. As described in said section, the tables are dropped if they exist before being created.
%************************sensor threads*************
\subsubsection{Sensor threads}
The next stage is to initiate the sensor threads. The purpose of the threads is to continuously acquire new data from the sensor, validate the data and insert it into the database. A time delay is incorporated in the threads to reduce the throughput, since the sensors are unlikely to change significantly in fractions of a second.
\newline
Threads are only started if the sensor is installed - this is defined in the configuration file. For more detail on the configuration file see configuration (\ref{configuration}).
\newline
The database continuously gets more data as Tiberius is running. Therefore it is necessary to start removing old transducer data as it will become less relevant the older it gets. To do this the number of readings which a sensor is to hold is set; the number of readings (rows) does not exceed this; therefore newest information starts overwriting the oldest information. 

% TODO: overwrite after so many values.

Below is a shortened version of the GPS thread.
\begin{lstlisting}[style=custompython]
while True:
  gps_data = gps.read_gps()
  if gps_data is not False:
    if gps_read_id < GPS_NUMBER_OF_READINGS:
   		ins.insert_gps_reading(self.poly, gps_read_id, gps_data)
    else:  # start updating results
   		gps_update_id = gps_read_id % GPS_NUMBER_OF_READINGS
    	up.overwrite_gps_reading(self.poly, gps_update_id, gps_data)
    gps_read_id += 1
    no_data_time = 0
\end{lstlisting}

\subsubsection{Actuator decorators}

The motor and arm are actuators rather than sensors, and therefore are employ a different technique to update the value; this technique is further described in the Decorators section (\cite{decoratordefinition})
 
Unlike the sensors, the actuator database should not be updated in a periodic manner. This is because the actuators don't change unless specified by the control software. Therefore decorators are placed on the functions that alter an actuators position.
\newline
Decorators dynamically alter the functionality of a function, method or class without having to directly use subclasses \cite{decoratordefinition}. A decorator function is called, whenever the function it is decorating is called. So for this implementation, whenever the function to move an actuator is called, the function to update the actuators location is called.
\newline
The three function to move the arm are:
\begin{lstlisting}[style=custompython]
def rotate_waist(self,change, angle=None):  
\end{lstlisting}
\begin{lstlisting}[style=custompython]
def move_shoulder(self, change, angle=None):  
\end{lstlisting}
\begin{lstlisting}[style=custompython]
def move_elbow(self, change, angle=None):  
\end{lstlisting}
Therefore each one of these function is decorated with the database arm update function. This function takes the new values of X, Y, Z, waist, elbow, shoulder and writes it to the database. 
\newline

\subsubsection{Validation}
The data validation is currently performed in the the sensor threads. Each thread gets its data from the appropriate driver and 

\subsubsection{Ultrasonics}
There is currently no verification on this data.

\subsubsection{LIDAR}
The lidar SDK contains it's own validity on the data it recieves. It was considered to remove readings in which the distance was below the radius of tiberius, however this was decided against since this would disregard any overhanging object. 
\newline
Unforunatly the lidar is not the highest component of tiberius and therefore does not have a full 360 degree view. Its's line of sight is obstructred by the directional antenna. Therefore the lidar reading which have angles which are affected by the antenna are removed; the rest of the lidar readings are inserted into the database. 

\subsubsection{GPS}
The gps is checked to see if it is usable. 
The usability function returns usable  if the gps has a fix, it has a position and that information is recent. 
the gps validity is set to false.

\subsubsection{Compass}
Heading is read of compass, and the 10 most recent readings are stored into an array. 
The standard deviation is taken of the last 10 readings. 
If the standard deviation > 10 then the last reading must be false. Typically this may be because the robot has neared at an ionized metal structure such as a steel support beam in a building. 
The compass validity is set to false.




