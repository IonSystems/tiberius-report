\section{Background Theory}
Various other messaging patterns exist, such as publish and subscript; these patterns were considered and then rejected in favour of the database model.

\subsubsection{Definition}
A database is central location in which information is stored \cite{databasedefinition}. The most prevalent approach is the relational database. A relational database is a collection of data items organized as a set of formally described tables, which can be accessed or manipulated \cite{relationaldatabasedefinition}.

\subsection{Types of database}
Various different database vendors were on offer. 
ideally be free 

\subsubsection{SQL}


\subsubsection{Polyhedra}
The benefit polyhedra holds over other database's namely SQL in this case is that it is an in-memory database. This means it can handle a high throughput and has a very short access time. Tiberius is a real time robot, which means it actions must occur with out a noticeable delay. 
\newline
Polyhedra uses TCP/IP for communication between modules and SQL for data access; because data access uses SQL it makes it easier for the development team to debug problems since some of the development team already had experience working with MySQL which uses the same commands. 

\subsubsection{Validation}
The intent is to validate all sensor and actuator data before being inserted into the database. This ensures all the data stored in the database is correct, and all other threads reading the database can trust it is error free. 
\newline
It is beneficial for the user and possibly in the future some control software to know if a sensor is producing invalid data. Therefore a data validation table exists within the database; which sets a sensor to false when it gives invalid data. 
This table is read by the diagnostics program, which operates a series of LEDs to give a physical indication of erroneous sensor data. To read more on the diagnostics LEDs see diagnostic section (\ref{diagnostics}).
The table can also be read by control software in the future. By collating all the sensor validity into one table, it allows the validity of all the sensors to be determined by one table query as opposed to a query for each sensor table.

