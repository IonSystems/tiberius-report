\section{Verification and Validation}

\subsection{Testing}
\subsubsection{Overview}
Tiberius's In-memory Database is a critical area of our architecture. Two decisions regarding the architecture were which RPi to use and which database to use. A script was created to measure the speed at which different configuration could carry out operations. This script created a database, inserted arbitrary values, updated the values, queried the values and then deleted the database. All of these functions were carried out 1000 times and then an average was produced for each of the following tasks.
\begin{itemize}
\item create
\item insert
\item update
\item query
\item drop table
\end{itemize}
The script was tested using the SQL database and using the polyhedra database to determine which version was superior. The test was to be run again on three different raspberry pi models - model 1 version B, model 2 version B and model 3 version B. The development team, however, did not have a model 2, and so the results for that section are omitted. 

\subsubsection{Results}
The results are in tabular form and can be found under appendix A.

\subsubsection{Analysis}
Polyhedra tends to outperform SQL in most operations. For create, insert, update and delete polyhedra will usually perform the task in less than a tenth of time it takes SQL. Query is the only operation in which SQL out performs polyhedra, although the difference is less significant as seen in the other operations.
\newline
Changing between Raspberry Pi had a small effect on the SQL database; the only operation which had a noticeable effect was query which was sped up. This differed from polyhedra database which benefited significantly with each change in Raspberry Pi model. Every operation other than delete had a reduction in time by at least one order of magnitude.
\newline
Considering the above analysis, the best composition for Tiberius 3 is to run polyhedra database on a raspberry pi 3 model B. 
\newline
After testing and analysis, it was noticed that the tests were flawed by overlooking a confounding variable; namely the SD card used. The raspberry pi's were running different classes of SD card which differed in access time. The effects should not have been significant on the polyhedra database since it is an in-memory database and therefore the data was held in RAM. The test could be carried out again using the same script and eliminating confounding factors but it will likely have the same outcome which are the results proving that the best composition is polyhedra running on a raspberry pi 3. 



















