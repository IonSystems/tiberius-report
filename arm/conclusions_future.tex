\section{Conclusions and Future Work}

\subsection{Stronger base section}
The current A-frame is not very rigid and really needs to be cross-braced to stop it from flexing. This would most likely fix the homing problems on the shoulder where the limit switch is not pressed because the frame twists and misses the switch.

\subsection{Autonomous Operation}
In the future the arm could be controlled autonomously using data from the Kinect 2 to move the arm to the correct location. Although a more suitable task would be to move it to a height above or near the point of interest and then switch to manual control for more precise positioning.

\subsection{More strength with different plastics}
The arm in its current state is very much a development version. The plastic parts are not printed at very high infill and hence are not the maximum strength they could be. Due to the difficulties of making bespoke gears out of metal different materials could be tested to replace the current PLA on the worm gears since they are the weak spot of the design.

\subsection{Multiple end effectors}
The end effector on the arm is attached using the freely rotating wrist joint. The wrist can be easily detached by removing 2 bolts and then sliding it off. This allows for more sections to be added to the arm in the future or for a powered wrist joint and different end effectors.
  