% Who we are and what we study
This report describes the development of an autonomous vehicle, Tiberius. The Tiberius development team currently comprises of six students studying MEng Computing and Electronics at Heriot-Watt University, Edinburgh.
\newline

% Background on tiberius
The Tiberius robot was first envisioned in 2005 by Dr Jim Herd, and initial development work was carried out as part of a MSc project that same year. At that time the mechanical structure of Tiberius was a modular articulated vehicle with two wheels per car, the idea being that each car would be electronically detachable at the articulation point to allow each car to move independently using self-balancing control loops. This design proved too difficult to implement mechanically, given the resources available at the time.
\newline

% What we did this year
Tiberius has been re-designed from a structural and software design perspective. The starting point for this year's development iteration was a four wheeled fixed-axle structure, with incomplete integration between hardware devices, and Tiberius' Python software. We now have a completely new mechanical design, with four independently steerable wheels, and suspension.
\newline

% Rapid prototyping
The project has benefited greatly from the latest rapid prototyping technologies - for mechanical prototyping as well as software prototyping. 3D printing was used heavily throughout the development cycle, allowing a complete mechanical prototype to be assembled within the limited six month time frame. A continuation of software development in Python meant that existing code could be re-used and re-factored into our new software suite.
\newline

% What this report contains
Contained within this report is an in-depth the design and implementation of Tiberius, as well providing relevant background theory. Reflections are made on the execution of the project from our point of view in our Conclusions, as well as descriptions of future areas of improvement and future work.

\pagebreak
