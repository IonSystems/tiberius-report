This report describes the plenary stages of the development of an autonomous vehicle, Tiberius. The Tiberius development team currently comprises of six students studying MEng Computing and Electronics at Heriot-Watt University, Edinburgh.
\newline
The Tiberius robot was first envisioned in 2005 by Dr Jim Herd, and initial development work was carried out as part of a MSc project that same year. At that time the mechanical structure of Tiberius was a modular articulated vehicle with two wheels per car. Each car would be electronically detachable at the articulation point to allow each car to move independently using self-balancing control loops. This design proved too difficult to implement mechanically, given the resources available at the time.
\newline
Tiberius, as it stands today after many development cycles, is a four wheel drive fixed-axle vehicle. The mechanical design of the original robot will not be modified further however the second vehicle design will have movable axles.  The result will be two mechanically different functional vehicles.
\newline
The plenary stage of this project involves:
\begin{itemize}
\item A literature review; describing the state of the art in the topic.
\item Project Integration; describing the responsibilities of each team member and their contributions.  This includes a detailed plan for the remaining development phase of the project, a preliminary proof of concept to back up planned work and a description of any work to be undertaken in a future project.
\item Project Risks and their Mitigation; a detailed risk assessment.
\end{itemize}
\pagebreak
