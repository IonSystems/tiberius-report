\chapter{Web Interface Request Code Walkthrough}
\label{app:web_walkthrough}
\pagestyle{cameron}

This walkthrough will provide a detailed description of all the steps that take place to communicate a command from the web interface to Tiberius. A standard configuration is assumed. A robotic arm command will be used as the example throughout this walkthrough.
\newline

\begin{enumerate}

\item The sequence of events is initiated by the user pressing a button on the web interface. In order to produce a button on the web page, the following \gls{HTML} code is used. See Code Listing \ref{html_button}. 
\newline

\begin{lstlisting}[caption=HTML Button, label=html_button]
<button id="arm_button_y_minus" class="btn btn-white" title="Y -">
    <span class="fa fa-2x fa-arrow-down"></span>
</button>
\end{lstlisting}

\item The below jQuery function is executed when the button is clicked. The send\_command function, with specific parameters for the button clicked. See Code Listing \ref{cl:html_listener}.
\newline

\begin{lstlisting}[language=JavaScript, caption=jQuery Click Listener, label=cl:button_listener]
$('#arm_button_y_minus').click(function() {
		send_command(ip_address, 'arm_dy', -d);
});
\end{lstlisting}

\item The send\_command function sends a \gls{HTTP} \gls{POST} request to the \textit{web server}, passing on the parameters into the request. Note that we cannot directly message Tiberius's API at this point, as we are client-side. See Code Listing \ref{cl:js_ajax_post}.
\newline

\begin{lstlisting}[language=JavaScript, caption=Javascript message to web server, label=cl:js_ajax_post]
function send_command(ip_address, command_name, command_value){
	$.ajax({
			url: '../send_arm_request',
			type: 'POST',
			data: {
				'command_name':command_name,
				'command_value':command_value,
				'ip_address':ip_address
			},
			success: function (result) {
				//alert("anything");
			},
			error: function(error_msg){
				alert(error_msg);
			}
	});
}
\end{lstlisting}

\item The Django web app forwards the \gls{POST} request (with a matching URL) to the send\_arm\_request view. See Code Listing \ref{cl:django_urls}.
\newline

%from web interface control urls.py
\begin{lstlisting}[style=custompython, caption=Django URL Dispatcher, label=cl:django_urls]
urlpatterns = [
  url(r'^$', views.index, name='index'),
  url(r'^(?P<id>\d+)/', views.control, name='control'),
  url(r'^send_control_request', views.send_control_request,
  name='send_control_request'),
  url(r'^send_task_request', views.send_task_request,
  name='send_task_request'),
  url(r'^send_arm_request', views.send_arm_request,
  name='send_arm_request'),
]
\end{lstlisting}

\item The send\_arm\_request view forwards the request over the \gls{VPN}. Because Javascript code is executed client-side, we have no guarantee that the particular client will be connected to the VPN, so all requests are relayed through the web server to guarantee access to the \gls{VPN}.

Some security measures are in place here, the request to the web server must be a \gls{POST} request and a valid authentication token must be present. See Code Listing \ref{cl:django_arm_view}.
\newline

  % From web interface control views file.
  \begin{lstlisting}[style=custompython, caption=Django Arm Request Forwarding View, label=cl:django_arm_view]
  @require_http_methods(["POST"])
  def send_arm_request(request):
      headers = {'X-Auth-Token': settings.SUPER_SECRET_PASSWORD}

      # Contruct url for motor resource on Control API
      ip_address = request.POST.get('ip_address')
      url_start = "http://"
      url_end = ":8000/arm"
      url = url_start + ip_address + url_end
      response = ""

      try:
          r = requests.post(url,
                            data=request.POST.lists(),
                            headers=headers)
          response = r.text
      except ConnectionError as e:
          response = e
      return HttpResponse(response)
  \end{lstlisting}

\end{enumerate}

