\section{Conclusions and Future Work}

\subsection{Future Work}
The web interface is very easy to extend, encouraging future development.
The work carried out this year is simply a proof-of-concept, so there is a large
chunk of work that could be done to further improve intergration with Tiberius.

\subsubsection{Joystick Integration}
During development of Tiberius - before all systems were fully integrated, we required an easy
to control Tiberius, this was realised by the development of a keyboard control script, allowing a developer to control the motors and robotic arm. This is suitable for testing, but it soon became apparent that a middle-ground interface was required that is self-sustained, like the keyboard control, but was also aesthetic and intuitive to use.

It was suggested that a console controller could be integrated, similar to an existing system used in the Oceans Laboratory. Development of a console controller interface should be initiated by talking to the Oceans Laboratory, they kindly offered a spare controller to be used on Tiberius.


\subsubsection{MPEG-2 Video Streams}
Time constraints during the project did not allow for the implementation of
\gls{MPEG-2} video streams from Tiberius to the web interface.

What is currently implemented is inefficient in terms of network bandwidth and
compute power on Tiberius' PC, and the device viewing the stream. Rather than streaming video, it was easier to rapidly publish snapshots from the webcam and Kinect camera. On the receiving end, a simple iframe automatically refreshes to load the next image.

We are aware of the inefficiencies, but haven't had the time to implement a full video stream. This is a suggested improvement if the video streaming is to be improved.

\subsubsection{Database Synchronisation}
During a full system run there are currently two seperate databases up and
running. The in-memory database runs on a Raspberry Pi (Tiberius-side) and the
SQLite database running on a web server. These two databases enable the
possibility of database synchronisation. This would ease the communication of
information between the web interface and Tiberius. Currently, communication of
data is achieved by sending HTTP requests to the Control API, where the
requests are translated to an SQL query, and the results of that query returned
in the HTTP response.

Database synchronisation would be an interesting alternative for communications,
but comes with benefits as well as drawbacks.

An Sqlite implementation of Tiberius's in-memory database exists, but isn't
favoured for normal use. Polyehedra demonstrated superior performance in
our benchmark testing. The Python database wrapper has been designed to allow
any SQL database engine to be used. Further information about the database
wrapper can be found in our online documentation.


\subsubsection{Improved Security}
A framework is already in place to support authenticated requests between
Tiberius and the web interface. However, currently we have a fixed
authentication token, configured via the web interface and API settings.

This is is an acceptable level of security for the scope of this project, which
is purely research and education-based work. However, security improvements
should be considered as a possible area of work for future years, if this
project is to ever be considered as a more commercial/educational project.

\subsubsection{Addition to PyPI}
It was originally conceived that our software would be in a suitable state for submission to \gls{PyPI}, by the end of the project. However, our focus was never concentrated on this goal, hence it was not reached. Some further work is required on the setup script, to ensure compliance with \gls{PyPI}'s recommendations.
This work includes the following:
\begin{itemize}
\item Remove all hard coded 'sudo' commands, and ensure the software does not rely on superuser privileges.
\item Refactor setup.py to make more use of the setuptools package, rather than using explicit commands that are not as portable, or reliable.
\item Increase unit testing coverage, this is bound to throw up issues that have not been thoroughly tested.
\item Introduce Sphinx docstrings in order to generate code documentation for readthedocs.io.
\end{itemize}

\subsubsection{Restoring support for Android}
An Android Application was developed in the year prior to ours, by Tasos Deligiannis. It was always our intention in re-integrate this application with our new Control API, however time did not allow. This would be an interesting task, that would require familiarisation with HTTP, and Android Programming.

Restoring support would involve changing the Android Application to interface with the Control API via HTTP POST requests. The control API already supports motor, database and robotic arm interaction. But does not currently support control of missions. The control of missions is another matter to be discussed elsewhere.

\subsubsection{Control API Mission Support}
