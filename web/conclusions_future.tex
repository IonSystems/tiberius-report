\section{Conclusions and Future Work}

\subsection{Future Work}
The web interface is very easy to extend, encouraging future development.
The work carried out this year is simply a proof-of-concept, so there is a large
chunk of work that could be done to further improve intergration with Tiberius.

\subsubsection{Joystick Intergration}

\subsubsection{MPEG-2 Video Streams}
Time constraints during the project did not allow for the implementation of
\gls{MPEG-2} video streams from Tiberius to the web interface.

What is currently implemeted is inefficient in terms of network bandwidth and
compute power on Tiberius's PC, and the device viewing

\subsubsection{Database Synchronisation}
During a full system run there are currently two seperate databases up and
running. The in-memory database runs on a Raspeberry Pi (Tiberius-side) and the
SQLite database running on a web server. These two databases enable the
possibity of database synchronisation. This would ease the communication of
information between the web interface and Tiberius. Currenly, communication of
data is acheieved by sending HTTP requests to the Control API, where the
requests are translated to an SQL query, and the results of that query returned
in the HTTP response.

Database synchronisation would be an interesting alternative for communications,
but comes with benefits as well as drawbacks.

An Sqlite implementation of Tiberius's in-memory database exists, but isn't
favoured for normal use. Polyehedra demonstarted superior performance in
our benchmark testing. The Pytohn database wrapper has been designed to allow
any SQL database engine to be used. Further information about the database
wrapper can be found in our online documentation.


\subsubsection{Improved Security}
A framework is already in place to support authenticated requests between
Tiberius and the web interface. However, currently we have a fixed
authentication token, configured via the web interface and API settings.

This is is an acceptable level of security for the scope of this project, which
is purely research and education-based work. However, security improvements
should be considered as a possible area of work for future years, if this
project is to ever be considered as a more commercial/educational project.
