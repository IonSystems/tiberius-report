\section{Design and Implementation}

\subsection{Subsystem Overview}
\label{sec:web_subsystem_overview}
A lightweight Python framework was used to create an API, the communication bridge between Tiberius and the \gls{webinterface}. The \gls{webinterface} interacts with Tiberius by sending authenticated \gls{HTTP} \gls{POST} requests to the \gls{API} running on Tiberius' \gls{controlpi}.
\newline
Each request is sent with an X-Auth \gls{POST} parameter, with a authentication token. The authentication token set on Tiberius and the web server must match.
\newline
The \gls{controlapi} and \gls{webinterface} server must be operating on the same network, to allow a route for packets to reach their destination address. In our instance we have used a \gls{VPN} to allow all devices to operate on the same network.
\newline
Diagram \ref{fig:api-modules} gives an overview of all \gls{webinterface} and \gls{controlapi} components.
\newline

% Diagram illustrating all the modules used within the API and web interface.
\begin{figure}[!htb]
\begin{center}
\includegraphics[width=14cm]{api_modules.png}
\end{center}
\caption{API Modules}
\label{fig:api-modules}
\end{figure}
\noindent
The following describes each web application component with reference to Figure \ref{fig:api-modules}:

\begin{enumerate}[label=(\alph*)]
  \item Tiberius API: Uses Falcon to accept \gls{HTTP} \gls{POST} requests and act upon the body of the messages.  % (a)
  
  \item Sensors: Provides access to sensor data, bypassing the database.  (GPS, Ultrasonics, Compass) % (b)
  
  \item Task Control: Contains functions to start, pause and stop tasks. % (c)
  
  \item Database: Controlled access to database. Mostly query functions to allow web interface to fetch the latest sensor data. % (d)
  
  \item Debug: Immediately sends a response to every request received, used to test response rate. % (e)
  
  \item Motor Control: Allows Tiberius's motors to be controlled from anywhere on the network. % (f)
  
  \item Navigation: If a valid request is received, a point-to-point algorithm is started, navigating Tiberius to the given GPS waypoint. % (g)
  
  \item HTTP POST: All requests are of the POST variety, this is slightly more secure that GET requests, because parameters do not appear in the browser URL and no caching, history, or bookmarking can be done.% (h)
  
  \item Tiberius Web Server: Runs on an external web server, although it can be deployed on any PC if resources do not allow access to a web server. % (i)
  
  \item Control: Contains web pages with control panels, allowing the user to control the motors and robotic arm. % (j)
  \item Fleet: Lists all robots registered on the website. % (k)
  \item Mission Planner: Allows a user to design a mission by plotting waypoints and assigning tasks to those waypoints. % (l)
  \item Dashboard: The front page of the web application, provides links to other useful resources, and introduces the user to the purpose of the web application. % (m)
  \item Users: Manages the users that have access to the web application. % (n)
\end{enumerate}

\subsection{Control API}
\label{sec:des_web_control_api}
\subsubsection{Overview}
The \gls{controlapi} is automatically started in the startup script (discussed in section \ref{sec:misc_startup_script}). Currently the \gls{controlapi} runs using a basic WSGI server implementation - \texttt{wsgiref.simple\_server} \cite{python-wsgiref-simple_server}.
\newline
As the \gls{controlapi} is written in \gls{python}, it integrates seamlessly with the existing code base. Appendix \ref{app:on_post_code} is an example resource, showing how motor control requests are handles upon arrival.
\newline
\texttt{Resource}s make up most of the \gls{controlapi} code base, with one middleware component to handle authentication.



\subsubsection{Resources}

The \gls{controlapi} is responsible for a lot of different subsystems. In order to provide access to all of the subsystems in a modular manner, each subsystem is represented by a separate \texttt{Resource}. Each resource implements an \texttt{on\_post()} \gls{method}, which contains the logic to deal with \gls{POST} requests. An example of such a \texttt{on\_post()} \gls{method} is described in depth in Appendix \ref{app:on_post_code}. Table \ref{tab:web_api_resources} described the function of each resource used.

% Table describing resources and their functions.
\begin{table}[!htb]
\centering
\begin{tabular}{|l|l|p{6cm}|}
\hline
\textbf{Resource}      & \textbf{URL} & \textbf{Description}                                                                   \\ \hline
SensorResource         & \texttt{/sensors}     & Provides access to sensor data, without using the database. (Not recommended)          \\ \hline
MotorResource          & \texttt{/motors}      & Allows control of four fixed axis motors. Direction and speed.                         \\ \hline
RobotArmResource       & \texttt{/arm}         & Allows control of the robotic arm, in cartesian form or individual joint manipulation. \\ \hline
DebugResource          & \texttt{/debug}       & Instantly replies to requests for response time measurements.                          \\ \hline
TaskControllerResource & \texttt{/task}        & Starts, pauses, resumes and stops tasks. Used by the web interface during missions.    \\ \hline
DatabaseResource       & \texttt{/database}    & Provides access to pre-defined database interactions.                                  \\ \hline
StatusResource         & \texttt{/status}      & Returns a generic response containing the current state of Tiberius.                   \\ \hline
NavigationResource     & \texttt{/navigation}  & Initiates navigation algorithms when passed valid parameters. (Point-to-point and A*)  \\ \hline
\end{tabular}
\caption{API resources and their functions}
\label{tab:web_api_resources}
\end{table}

\subsubsection{Authentication Middleware}
Each request that the \gls{controlapi} receives is passed through an authentication filter, passing only valid authenticated requests. The implemented authentication mechanism looks for a valid token in the \texttt{X-Auth-Token} parameter of the request. Appendix \ref{app:authentication_middleware} includes the \gls{python} authentication middleware, with a description.
\newline
Figure \ref{fig:api-auth-middleware} illustrates the authentication middleware filtering out invalid (unauthenticated) requests, and passing authenticated requests as if nothing happened. A walkthrough of the sequence of events to send an authenticated \gls{HTTP} \gls{POST} request to the \gls{controlapi} is described in Appendix \ref{app:web_walkthrough}.

% Diagram illustrating all the modules used within the API and web interface.
\begin{figure}[!htb]
\begin{center}
\includegraphics[width=12cm]{api_middleware.png}
\end{center}
\caption{API Authentication Middleware}
\label{fig:api-auth-middleware}
\end{figure}
\noindent
The authentication token is set in the \gls{controlapi} and \gls{webinterface} configuration files.

\subsection{Web Interface}

\subsubsection{Overview}
Many independent applications exist on the \gls{webinterface}, allowing users to run missions and teleoperate robots. The necessary tools exist on the interface to create/edit/remove robots, tasks and missions.
\newline
An overview of the layout and structure of the web interface's pages is best described in a table (Table \ref{tab:web_pages}). Many other pages exist to carry out rudimentary tasks, but are of no significance. Screenshots of the web interface can be viewed in Appendix \ref{app:web_screenshots}.

% Table describing key web interface pages.
\begin{table}[!htb]
\centering
\begin{tabular}{|l|l|p{7cm}|}
\hline
\textbf{Application} & \textbf{URL} & \textbf{Description} \\ \hline
\multirow{6}{*}{Mission Planner} & /missionplanner                        & View all missions in a table.                                                                              \\ \cline{2-3} 
                                 & /create                                & Define basic parameters for a new mission.                                                                 \\ \cline{2-3} 
                                 & /plotting                              & Plot the waypoints to follow, as part of the mission. Assign tasks to the waypoints.                       \\ \cline{2-3} 
                                 & /manage\_tasks                         & View all tasks in a table                                                                                  \\ \cline{2-3} 
                                 & /view\_mission/\textless id\textgreater & Displays all associated data with mission: supported platforms, tasks, waypoints.                          \\ \cline{2-3} 
                                 & /view\_task/\textless id\textgreater    & Displays description of the task, and it's supported platforms.                                            \\ \hline
\multirow{3}{*}{Fleet}           & /fleet                                 & Displays all robots in the fleet. A fleet is a collective term used for robots.                            \\ \cline{2-3} 
                                 & /modify/\textless id\textgreater        & Modify an existing robot.                                                                                  \\ \cline{2-3} 
                                 & /view/\textless id\textgreater          & View an existing robot.                                                                                    \\ \hline
Control                          & /control/\textless id\textgreater       & The user can operate a robot remotely (teleoperate), with control of the drive motors and the robotic arm. \\ \hline
\end{tabular}
\caption{Key web interface pages}
\label{tab:web_pages}
\end{table}

\subsubsection{Mission Planning}
\label{web-missions} %This needs to stay, stuart ref section this from nav :}
\label{sec:web_design_missions}

One of the primary purposes of the web interface is to allow users to design their own missions. The user can then trigger the start of a mission and observe, as Tiberius works through the list of waypoints and tasks.
\newline
The waypoint-orientated design of missions was introduced to provide a strong location-based foundation for missions. Tasks were incorporated to support non-navigational objectives.
\newline
In previous years, missions have involved recognising objects and navigating past obstacles. This year, missions are more explicitly defined by a user, via the web interface which involves the following:

\begin{enumerate}
\item Give the new mission a name and description.
\item Plot waypoints, defining the path that Tiberius must take to complete the mission.
\item Assign tasks to waypoints, these tasks are started once Tiberius has reached it's waypoint.
\end{enumerate}

Missions are stored in a database on the web server, and sequentially transmitted to Tiberius when a mission is started. Tasks can also me initiated manually at any time, from the web interface. Available tasks are defined in Tiberius's \gls{controlapi} software. Currently the amount of implemented tasks is slim, but a strong software infrastructure is in place to define new tasks. Code listing \ref{code:drive-task} is an example of a simple task.

\begin{lstlisting}[style=custompython,label=code:drive-task][!htb]
class DriveForwardsUntilWall(Task):

    def __init__(self):
        super(DriveForwardsUntilWall, self).__init__()
        self.task_name = 'Drive Forwards Until Wall'
        self.task_id = 0
        self.task_description = '''Drive forwards at half speed until a wall
				is sensed using the ultrasonics with a
				distance of 10cm.'''

    def runTask(self):
        self.control.driveForwardUntilWall(50)
        self.completeTask()

    def pauseTask(self):
        self.control.stop()

    def resumeTask(self):
        self.runTask()
\end{lstlisting}
\noindent
Referring to Code Listing \ref{code:drive-task}, each task implements \texttt{runTask()}, \texttt{pauseTask()} and \texttt{resumeTask()}. These functions are called from the \gls{controlapi} when an appropriate task request is received. The \texttt{task\_id} is used by the \gls{controlapi} to distinguish between tasks. The web interface defines all known tasks in it's database. The \gls{webinterface} uses a fixture to populate the database with tasks \cite{github-fixtures}.


\subsubsection{Teleoperation}

The web interface allows tele-operation of Tiberius's motors and robotic arm. Tiberius can be tele-operated during or outwith missions. Both the motors and robotic arm can be controlled from the same page. The interface supports motor speed control, skid steering and forward and backwards motion. Each axis of the robotic arm can also be controlled. A live video stream from a webcam allows Tiberius to be teleoperated, without the user having a visual line of sight.
\newline
Future improvements to our current tele-operation implementation are discussed in section \ref{sec:web_future_work_access_tele}.


\subsubsection{Fleet Management}
Within the \gls{webinterface}, the user can store details about all robots. The most important detail to store is the device's IP address, this is used by the web interface to send requests to the correct device. Figure \ref{fig:web_fleet} in Appendix \ref{app:web_screenshots} is a screenshot of the web interface, displaying the fleet used throughout development.