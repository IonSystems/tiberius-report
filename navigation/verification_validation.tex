\section{Verification and Validation}
In order to confirm that the navigation systems where working as expected, they needed to be verified. This meant that the algorithms and sensors used to create the navigation portion of the robot, real world testing was required to determine the effectiveness of the systems.
\subsection{Point to Point}
As a means to test the point to point algorithm the old version of tiberius was used so that the \gls{GPS} and movement could be check outdoors. This meant that the movement connected to the algorithm could be tested while the newer mode \gls{tiberius3} was still in development. The code went through multiple iterations before finally performing as expected. The testing itself consisted of taking a \gls{GPS} location and then moving the robot to a different location so that it could return to the original position, or within a meter. This test seemed to successfully move from one location to the given destination, after some more code changes. The test was run via a manual script that called the point to point function \ref{} whereas the final implementation should be called from the API through the web interface.
\subsection{A-Star}
With the A-Star algorithm there was some testing of the grid creation and path finding, however it was not fully tested on the robot due to limited time. The grid seem to successfully create with the correct latitude and longitude positions, the positions are separated by a distance of one meter, where each cell represents a one meter squared area. This means that tiberius will always be inside a cell, without having to be in the exact centre of said cell.
\subsection{Sensor Data}
The main sensors related to the current navigation systems that have been tested where the \gls{GPS} and the compass. These two sensors are vital components for the outdoor movement algorithms. Testing within the lab proved to be difficult as the compass ran into large amounts of interference when tiberius II moved up the room. This meant that the tests showed that the compass was giving incorrect data however it was due to the environment rather than the coding. The \gls{GPS} was also difficult to test within the lab as it would struggle to find a fix indoors. 