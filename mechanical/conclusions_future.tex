\section{Conclusions and Future Work}
The mechanical redesign of Tiberius has been an ongoing process, but the result speaks for itself; while recognisable as a logical progression from the previous version of the robot, this new iteration has evolved dramatically. It has many new features while retaining the advantages of the old one, and although some work still needs to be done on the steering system, the infrastructure is in place for a future group to take over.
\newline
\subsection{Future Development}
In hindsight there are some small areas where the mechanics could be improved. The wishbones for example are attached directly onto the chassis; ideally these should be attached via bearings to allow for smoother and sturdier movement. This could be done without altering the current design, simply by adding rotational bearings between the wishbones and chassis.
\newline
The wishbones themselves, as well as the steering knuckle, are made from laser-cut \gls{Acetal} plastic. This was a design choice in order to save time; strong enough for the immediate future (and much stronger than printed PLA, which proved inadequate), but not ideal. Given more time these parts would have been made from metal sheet. This is an upgrade which can be made in the future.
\newline
Overall the evolution from Tiberius II to Tiberius III has been a success, and the mechanics are fully in place for further software development in the future.