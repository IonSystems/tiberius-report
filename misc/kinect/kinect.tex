\section{Kinect 2}

\subsection{Introduction}

Picture of depth grid

\subsection{Standard Deviation}\label{StandardDeviation}

\subsection{Chunking}\label{Chunking}
Chunking is the process of dividing a large dataset into several smaller ones. In our case this allowed us to split the depth image data that was received from the Kinect 2 and splitting it into an 8 by 8 grid. We could then perform the calculations on each element of the grid. The main reason for chunking our data was give us a much lower resolution image which was more useful and easily used on the control pi.

diagram of structure of camera streamer

\subsection{Camera Streamer}
This C\# console application runs on the pc. It takes all the data coming from the Kinect 2 and processes it into a useful format.
The depth data is processed using the methods described in \ref{Chunking} \textit{Chunking} and \ref{StandardDeviation} \textit{Standard Deviation}. The colour, infrared and depth streams are all downscaled to a lower resolution and then saved as JPEG image files on disk. Windows's built in web server IIS (Internet Information Services) then serves the image files to anyone that requests them.
  
refrence picture of website view

\pagestyle{euanstuart}













