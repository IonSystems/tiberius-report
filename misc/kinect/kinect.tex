\section{Kinect 2}

\subsection{Introduction}

Picture of depth grid

\subsection{Standard Deviation}\label{StandardDeviation}
The Kinect provides a range of useful data that can be used to created a wide range of data for different proposes. An example of this is to use the depth data to create a standard deviation grid that can be used for the A-Star algorithm within navigation for \gls{tiberius3}. This grid takes in the depth data that can be obtained from the Kinect and splits it into chunks as described in the next section \ref{Chunking} \textit{Chunking}. The data is then modified using the standard deviation equation that is seen in (figure of equation or latex equation?). The result can be seen in (the depth grid figure), a large pixel recreation of the depth data has been created. This can then be mapped to cells within the A-Star grid when performing the navigation algorithm in order to gain information about walls or terrain in front of \gls{tiberius3}.
\subsection{Chunking}\label{Chunking}
Chunking is the process of dividing a large dataset into several smaller ones. In our case this allowed us to split the depth image data that was received from the Kinect 2 and splitting it into an 8 by 8 grid. We could then perform the calculations on each element of the grid. The main reason for chunking our data was give us a much lower resolution image which was more useful and easily used on the control pi.

diagram of structure of camera streamer

\subsection{Camera Streamer}
This C\# console application runs on the pc. It takes all the data coming from the Kinect 2 and processes it into a useful format.
The depth data is processed using the methods described in \ref{Chunking} \textit{Chunking} and \ref{StandardDeviation} \textit{Standard Deviation}. The colour, infrared and depth streams are all downscaled to a lower resolution and then saved as JPEG image files on disk. Windows's built in web server IIS (Internet Information Services) then serves the image files to anyone that requests them.
  
refrence picture of website view

\pagestyle{euanstuart}













