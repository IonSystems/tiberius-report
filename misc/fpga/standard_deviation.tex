\section{FPGA Standard Deviation}
During the development of \gls{tiberius3} there was some consideration as to the best way to develop the data coming from the kinect. Original the data was going to be passes to a GPU that would have been connected to the PC. This however would have used a lot of power while running. Therefore the decision was made to replace it with an \gls{FPGA} that would communicate over Ethernet with the PC to calculate results for any section of data that might have been implemented. 
\subsection{Ethernet communication}

\subsection{Standard Deviation Implementation}

\subsection{Limited Gains compared to Cost}
The use of an \gls{FPGA} could have a positive effect on the speed of calculating the standard deviation of the kinect depth data, however it comes at to high a cost. This cost is not simply monitory as the increased power consumption and the need for very fast Ethernet communication meant that it was not a feasible addition to \gls{tiberius3}. The main flaw in the idea of using an \gls{FPGA} as a unit for calculating the standard deviation was that the data needed to be passed between the PC and \gls{FPGA} multiple times before a result could be entered into the database. With the \gls{FPGA} board's available within the project the best Ethernet that could be obtained was well below a viable level for a speed-up to be gained compared to simply using the CPU of the PC.

\pagestyle{stuart}













