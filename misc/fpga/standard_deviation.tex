\section{FPGA Standard Deviation}
During the development of \gls{tiberius3} there was some consideration as to the best way to process the data coming from the Kinect 2. Originally the data was going to be passed to a GPU that would have been connected to the PC. This however would have used a lot of power while running. Therefore the decision was made to replace it with an \gls{FPGA} that would communicate over Ethernet with the PC to calculate results for any section of data that might have been implemented. As two members of the group owned their own cyclone V GX boards \cite{CycloneV} it was decided that one could be used to demonstrate the concept during the current iteration of Tiberius' development so that a custom board could be created in a later project. For the proof of concept of the \gls{FPGA}, the standard deviation calculation was used to create a grid of the terrain in front of Tiberius.
\subsection{Ethernet communication}
In order to communicate between the \gls{FPGA} and the PC, Ethernet was required. This meant that a Ethernet module was required for both send and receive. An implementation of both send and receive was found on FPGA4Fun.com \cite{FPGA4Fun} where the \gls{FPGA} could send data out without any external hardware, however the receive module required an external circuit in order to correctly receive data from an external source. The resulting code only produced a data rate of 10Mbits/s which meant that the data would take a long time to be transferred between the PC and \gls{FPGA} before and after the calculation had been done. The send module was tested and successfully sent Ethernet packets to the test IP address. Receive was never tested due to the decision to remove the \gls{FPGA} from \gls{tiberius3} and as a result the external circuit was also never made.
\subsection{Standard Deviation Implementation}
The actual standard deviation implementation was similar to the final C\# implementation in that it used the same equation to process the data. However the \gls{FPGA} version would have been created to calculate data in parallel which would have increased the speed of calculation compared to the CPU on the PC. The equation can be seen in figure \ref{SDEquation}. The data would have been sent to the \gls{FPGA} in chucks so to create the grid that was in the end created using the c\# code.
\subsection{Limited Gains compared to Cost}
The use of an \gls{FPGA} could have a positive effect on the speed of calculating the standard deviation of the Kinect 2 depth data, however it comes at too high a cost. This cost is not simply monitory as the increased power consumption and the need for very fast Ethernet communication meant that it was not a feasible addition to \gls{tiberius3}. The main flaw in the idea of using an \gls{FPGA} as a unit for calculating the standard deviation was that the data needed to be passed between the PC and \gls{FPGA} multiple times before a result could be entered into the database. With the \gls{FPGA} board's available within the project the best Ethernet that could be obtained was well below a viable level for a speed-up to be gained compared to simply using the CPU of the PC. This was a unsuccessful development path that took a good amount of time however the correct decision was made to not include an \gls{FPGA} in this version of the Tiberius robot. It is something that could be looked into in the future via a FPGA which is PCIExpress compatible.

\pagestyle{stuart}













