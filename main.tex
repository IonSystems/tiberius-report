% LaTeX cheatsheet at:
% https://docs.google.com/document/d/1siJmv6TOJMdM7CNSkKBk8bU9iwQ3qLodOJSmk4UGuCc/edit?usp=sharing

%% This example file demonstrating the use of the Caltech Thesis LaTeX Class file
%% has been provided by Overleaf to help you get started writing your thesis.
%%
%% This example uses the new_cit_thesis.cls file produced by Daniel M. Zimmerman
%% An updated version of this class file has been produced by Ling Li, and is
%% available at: https://www.overleaf.com/read/njjxyzszvjzx
%%
%% Before starting your thesis, it is recommended to read the Caltech Library
%% thesis guide at: http://libguides.caltech.edu/content.php?pid=22955&sid=487035
%%
\documentclass{new_cit_thesis}

%Definition of our names
\newcommand{\cameron}{Cameron A. Craig}
\newcommand{\euan}{Euan W. Mutch}
\newcommand{\duncan}{Duncan A. Robertson}
\newcommand{\stuart}{Stuart J. Thain}
\newcommand{\aidan}{Aidan P.J. Gallagher}
\newcommand{\andrew}{Andrew J. Rigg}

% Define figure folders, so we don't need to define the full path
% just the figure name.
\usepackage{graphicx}
\graphicspath {
	{system_overview/figures/}
	{mechanical/figures/}
    {web/figures/}
    {communication/figures/}
    {power/figures/}
    {arm/figures/}
    {navigation/figures/}
}

\renewcommand{\arraystretch}{1.5} %Set a bit more padding in the tables

\usepackage[a4paper, total={6in, 8.5in}]{geometry}
\usepackage[absolute]{textpos} % Used for positioning logo absolutely
\usepackage{url}				%Use package for generating urls
\usepackage{multirow}			%For authors table
\usepackage[table,x11names]{xcolor} %To give table rows a bit of colour if so desired
\usepackage[export]{adjustbox}	%To position logo on right
\usepackage{xstring}			%For case statement on strings
\usepackage[toc,page]{appendix}	%Use package for generating appendices
\usepackage{amssymb}			%Use package for checkmark

% Add support for equation captions
\usepackage{caption}
\DeclareCaptionType{capequ}[][List of equations]
\captionsetup[capequ]{labelformat=empty}

\usepackage{pdfpages}			%Use package for including pdf files
\usepackage{longtable}			%Used for tables over more than one page
\usepackage{booktabs}			%For height of stuff (adjustments)
%\usepackage{morefloats}		%Get more floats for stuff
%\extrafloats{100}				%same as above
\usepackage{nth}				%For adding nth st etc to numbers, pretty cool
\usepackage{cite}				%For references
\usepackage{textcomp}			%To suppress warnings from gensymb
\usepackage{gensymb}				%For special characters
\newcommand\mathplus{+}			%For math plus symbol in biblio
\usepackage{float}				%For positioning things correctly

\usepackage{listings}			%For code formatting and highlighting

% Define custom fomratting style for Python code
\lstdefinestyle{custompython}{
  belowcaptionskip=1\baselineskip,
  breaklines=true,
  frame=single,
  xleftmargin=\parindent,
  language=Python,
  showstringspaces=false,
  basicstyle=\footnotesize\ttfamily,
  keywordstyle=\bfseries\color{green!40!black},
  commentstyle=\itshape\color{purple!40!black},
  identifierstyle=\color{blue},
  stringstyle=\color{orange},
  numbers=left
}

% Add formatting support for Javascript
\lstdefinelanguage{JavaScript}{
  keywords={typeof, new, true, false, catch, function, return, null, catch, switch, var, if, in, while, do, else, case, break},
  keywordstyle=\color{blue}\bfseries,
  ndkeywords={class, export, boolean, throw, implements, import, this},
  ndkeywordstyle=\color{darkgray}\bfseries,
  identifierstyle=\color{black},
  sensitive=false,
  comment=[l]{//},
  morecomment=[s]{/*}{*/},
  commentstyle=\color{purple}\ttfamily,
  stringstyle=\color{red}\ttfamily,
  morestring=[b]',
  morestring=[b]"
}

\lstset{
%    language=JavaScript,
   backgroundcolor=\color{lightgray},
   extendedchars=true,
   basicstyle=\footnotesize\ttfamily,
   showstringspaces=false,
   showspaces=false,
   numbers=left,
   numberstyle=\footnotesize,
   numbersep=9pt,
   tabsize=2,
   breaklines=true,
   showtabs=false,
   captionpos=b,
   frame=single,
   xleftmargin=\parindent,
}


\usepackage{enumitem}			%Custom enumeration of lists
\setlist{nolistsep}             %Remove the massive spaces between bullets

\usepackage{mathtools}
\usepackage[normalem]{ulem}
\usepackage{fancyhdr}			%For the header names at the top right
\usepackage{tikz}				%For the logo at top of title page
\usepackage{notoccite}			%Stops citation numbering starting from their appearance in ToC
\usepackage{subcaption}			%For subfigures
\renewcommand{\headrulewidth}{0pt}
\setlength{\headheight}{14pt}

\fancypagestyle{cameron}{
        \fancyhead[R]{\cameron}
}
\fancypagestyle{cameroneuan}{
        \fancyhead[R]{\cameron \euan}
}
\fancypagestyle{euan}{
        \fancyhead[R]{\euan}
}
\fancypagestyle{duncan}{
        \fancyhead[R]{\duncan}
}
\fancypagestyle{stuart}{
        \fancyhead[R]{\stuart}
}
\fancypagestyle{aidan}{
        \fancyhead[R]{\aidan}
}
\fancypagestyle{andrew}{
        \fancyhead[R]{\andrew}
}

\fancypagestyle{andreweuan}{
		\fancyhead[R]{\andrew, \euan}
}

\fancypagestyle{blank}{
%This is blank
		\fancyhead[R]{ }
}

\newcommand{\getname}[1]{%
    \IfEqCase{#1}{%
        {cameron}{\cameron}%
        {euan}{\euan}%
        {duncan}{\duncan}%
        {stuart}{\stuart}%
        {aidan}{\aidan}%
        {andrew}{\andrew}%
    }[\PackageError{getname}{Undefined option to tree: #1}{}]%
}%



\newcommand\authordescription[2]{
  \begin{tabular}{lp{4cm}}
  \multirow{2}{*}{\includegraphics[width=2cm]{profiles/#1}} & \multicolumn{1}{l}{\getname{#1}} \\ 
                         & #2\\ 
  \end{tabular}
  \vspace*{0.5cm}
}				%Contains custom macros
\usepackage{csquotes}			%For big quotes

\renewcommand{\lstlistingname}{Code Listing}

%Row counting to continue list numbering across lists
\usepackage{array}
\newcounter{rowcount}
\setcounter{rowcount}{0}

\usepackage[xindy]{glossaries}
\makeglossaries
\newglossaryentry{domain-knowledge}{%
  name={domain knowledge},%
  description={valid knowledge used to refer to an area of human endeavour, an autonomous computer activity, or other specialized discipline}}


%*********************************************************************
%******************** WebBoy******************* **********************
%*********************************************************************
\newglossaryentry{mini-ATX}{
	name={mini-ATX},
    description={Standard motherboard form factor, measuring 17cm x 17cm}
}

\newglossaryentry{python}{
	name={Python},
    description={is a duck typed programming language with a efficient and readable syntax}
}

\newglossaryentry{teleoperation}{
	name={teleoperation},
    description={Remotely operating a device from a distance}
}

\newglossaryentry{waypoint}{
	name={waypoint},
    description={Geographical location that must be visited in order to complete a journey}
}

\newglossaryentry{imdb}{
	name={in-memory Database},
    description={Database management system that resides in main memory}
}
\newglossaryentry{ultrasonicsensor}{
	name={ultrasonic sensor},
    description={is a rangefinder capable of sensing distances between 0 and }
}
\newglossaryentry{RPi}{
	name={Raspberry Pi},
    description={is the first version of the Raspberry Pi, developed by the Raspberry Pi Foundation}
}
\newglossaryentry{RPi2}{
	name={Raspberry Pi 2},
    description={is the second version of the popular Raspberry Pi computer, developed by the Raspberry Pi Foundation}
}
\newglossaryentry{kinect}{
	name={Xbox Kinect},
    description={is a set of sensors designed to sense body position and gestures, that is also used in robotics to sense an indoor environment}
}
\newglossaryentry{microcontroller}{
	name={microcontroller},
    description={is is a small computer packaged in a single integrated chip}
}
%*********************************************************************
%*************************** MechMaster ******************************
%*********************************************************************
\newglossaryentry{citadel}{
	name={citadel},
    description={Tiberius's central power bank}
}

\newglossaryentry{armature}{
	name={armature},
    description={Rotating coil or coils of an electric motor}
}

\newglossaryentry{epicyclic gearbox}{
	name={epicyclic gearbox},
    description={Gear system consisting of one or more outer `planet' gears, revolving about a central `sun' gear.}
}

\newglossaryentry{holonomic}{
	name={holonomic},
    description={A robot is holonomic if it can move freely in all degrees in 2D space}
}
%*********************************************************************
%*************************** NavLord *********************************
%*********************************************************************
\newglossaryentry{MCL}{
	name={Monte Carlo Localization},
    description={Algorithm for robots to localize using a particle filter}
}
\newglossaryentry{KalmanFilter}{
	name={kalman filter},
    description={Algorithm that uses a series of measurements, containing statistical noise and other inaccuracies, and produces estimates of unknown variables}
}
\newglossaryentry{compass}{
	name={compass},
    description={Instrument used to to tell the orientation relative to the magnetic field of the Earth}
}
\newglossaryentry{hectorSLAM}{
	name={Hector SLAM},
    description={SLAM library developed as part of ROS}
}

%*********************************************************************
%*************************** Random Crap *****************************
%*********************************************************************
\newglossaryentry{canbus}{
	name={CAN Bus},
    description={Hardware communications protocol}
}
\newglossaryentry{android}{
	name={Android},
    description={is a linux-based mobile device operating system}
}
\newglossaryentry{RS232}{
	name={R2-232},
    description={is a serial communication standard that is popular within embedded systems}
}
\newglossaryentry{i2c}{
	name={I\textsuperscript{2}C},
    description={Hardware communications protocol}
}
\newglossaryentry{wifi}{
	name={WiFi},
    description={Wireless communications protocol defined in IEEE 802.11 standard.}
}
\newglossaryentry{mbed}{
	name={mbed},
    description={Cortex M3 development board developed by ARM\textregistered}
}
\newglossaryentry{LiFe}{
	name={$LiFePO_{4}$},
    description={Chemical formula for a Lithium battery technology.}
}
\newglossaryentry{LiPo}{
	name={$LiPO$},
    description={Chemical formula for a Lithium battery technology.}
}
\newglossaryentry{tiberius2}{
	name={Tiberius II},
    description={is the second development iteration of the Tiberius robot}
}
\newglossaryentry{ethernet}{
	name={Ethernet},
    description={is a link-layer protocol for TCP/IP}
}
\newglossaryentry{tiberius3}{
	name={Tiberius III},
    description={is the second development iteration of the Tiberius robot}
}
\newglossaryentry{infrared}{
	name={infrared},
    description={is a band of the electromagnetic spectrum having a wavelength just greater than that of the red end of the visible light spectrum but less than that of microwaves}
}
%*********************************************************************
%**************************** CommsKing ******************************
%*********************************************************************
\newglossaryentry{4g}{
	name={4G},
    description={Fourth generation of mobile communications technology}
}
%*********************************************************************
%**************************** Acronymns ******************************
%*********************************************************************
\newacronym{API}{API}{Application Program Interface}
\newacronym{LIDAR}{LIDAR}{Light Detection And Ranging}
\newacronym{MVC}{MVC}{Model View Controller}
\newacronym{VPN}{VPN}{Virtual Private Network}
\newacronym{GPS}{GPS}{Global Positioning System}
\newacronym{SLAM}{SLAM}{Simultaneous Localization and Mapping}
\newacronym{LDR}{LDR}{Light Dependant Resistor}
\newacronym{IEEE}{IEEE}{Institute of Electrical and Electronic Engineers}
\newacronym{INS}{INS}{Inertial Navigation System}
\newacronym{USB}{USB}{Universal Serial Bus}
\newacronym{SSD}{SSD}{Solid State Drive}
\newacronym{TCPIP}{TCP/IP}{Transmission Control Protocol/Internet Protocol}
\newacronym{XML}{XML}{eXtensible Markup Language}
\newacronym{JSON}{JSON}{JavaScript Object Notation}
\newacronym{DOM}{DOM}{Document Object Model}
\newacronym{DSN}{DSN}{Deep Space Network}
\newacronym{LED}{LED}{Light Emitting Diode}
\newacronym{DC}{DC}{Direct Current}
\newacronym{RPM}{RPM}{Revolutions Per Minute}
\newacronym{SPI}{SPI}{Serial Peripheral Interface}
\newacronym{DLR}{DLR}{Deutsches Zentrum für Luft- und Raumfahrt e.V. (German Aerospace Center)}
\newacronym{ROS}{ROS}{Robot Operating System}
\newacronym{OS}{OS}{Operating System}
\newacronym{PAT}{PAT}{Portable Appliance Testing}
\newacronym{HTTP}{HTTP}{Hyper-Text Transfer Protocol}
\newacronym{UGV}{UGV}{Unmanned Ground Vehicle}
\newacronym{UAV}{UAV}{Unmanned Aerial Vehicle}
\newacronym{AUV}{AUV}{Autonomous Underwater Vehicle}
\newacronym{ROV}{ROV}{Remotely Operated Vehicle}
\newacronym{SSE}{SSE}{Scottish and Southern Energy}
\newacronym{BLDC}{BLDC}{Brushless DC}
\newacronym{RADAR}{RADAR}{RAdio Detection and Ranging}
\newacronym{UV}{UV}{ultraviolet}
\newacronym{GGA}{GGA}{Global Positioning System Fix Data}
\newacronym{NMEA}{NMEA}{National Marine Electronics Association}
\newacronym{SPAD}{SPAD}{Single Photon Avalanche Detector}
\newacronym{SDK}{SDK}{Software Development Kit}


\begin{document}
\title{Final Group Report for the\\Development of an Autonomous Robot}
\author{
 \cameron	\\
  \aidan	\\
  \euan		\\
  \andrew	\\
  \duncan	\\
  \stuart	\\
}

\degreeaward{MEng Computing and Electronics}	% Degree to be awarded
\university{Heriot-Watt University}	% Institution name
\address{Riccarton, Edinburgh}		% Institution address
\unilogo{HW_logo}					% Institution logo
\tiblogo{tib_logo}					% Tiberius logo
\copyyear{\the\year}				% Year on diploma
\pubnum{}							% Publication number
\maketitle

\begin{frontmatter}

\begin{authors}



\begin{tabular*}{\textwidth}{l r}
%\centering
    \authordescription{cameron}{\url{cac30@hw.ac.uk}} & \authordescription{aidan}{\url{apg30@hw.ac.uk}} \\
    \authordescription{euan}{\url{em232@hw.ac.uk}} & \authordescription{andrew}{\url{ar339@hw.ac.uk}} \\
    \authordescription{duncan}{\url{dar30@hw.ac.uk}} & \authordescription{stuart}{\url{st298@hw.ac.uk}} \\
\end{tabular*}


\end{authors}

\begin{acknowledgements}
\begin{description}
% 
\item [Dr. James Herd] has provided us with the tools and components required to further develop Tiberius. Our thanks go out to him for his continued enthusiastic support throughout the year.

\item [Andrew Haston] worked with us during the second semester to realise our designs into actual hardware. He has worked with us in designing Tiberius' power distribution, and taken care of the manufacturing of PCB's and related paraphernalia.

\item [Dr. David Lane CBE FREng FRSE] has provided us with opportunities to showcase our work within the department, and he has been enthusiastic about the project throughout.

\item [Yvan Pettilot] has kept an eye on the progress of the project throughout the academic year, and provided some constructive advice regarding weatherproofing and control methods.

% This is a bad idea - no it is not
\item [Linus Torvalds] despite not knowing about the project or interacting directly with the team has provided the infrastructure which allowed the team to collaborate their code and run their code with his creation of git and the Linux kernel.

% \item [Stevie] has provided top-notch service and done all he could in terms of equipment provision.

\end{description}

\end{acknowledgements}

\begin{quote}
\vspace*{10cm}
\centering
%old one, we thought we would change to the other quote though since it was the original and less known (obviously we are super hipsters!)
%If I have seen further, it is by standing on %the shoulders of giants. \\- Isaac Newton

We are like dwarfs sitting on the shoulders of giants. We see more, and things that are more distant, than they did, not because our sight is superior or because we are taller than they, but because they raise us up, and by their great stature add to ours. \\ - John of Salisbury
\end{quote}

\begin{abstract}
This report describes the plenary stages of the development of an autonomous vehicle, Tiberius. The Tiberius development team currently comprises of six students studying MEng Computing and Electronics at Heriot-Watt University, Edinburgh.
\newline
The Tiberius robot was first envisioned in 2005 by Dr Jim Herd, and initial development work was carried out as part of a MSc project that same year. At that time the mechanical structure of Tiberius was a modular articulated vehicle with two wheels per car. Each car would be electronically detachable at the articulation point to allow each car to move independently using self-balancing control loops. This design proved too difficult to implement mechanically, given the resources available at the time.
\newline
Tiberius, as it stands today after many development cycles, is a four wheel drive fixed-axle vehicle. The mechanical design of the original robot will not be modified further however the second vehicle design will have movable axles.  The result will be two mechanically different functional vehicles.
\newline
The plenary stage of this project involves:
\begin{itemize}
\item A literature review; describing the state of the art in the topic.
\item Project Integration; describing the responsibilities of each team member and their contributions.  This includes a detailed plan for the remaining development phase of the project, a preliminary proof of concept to back up planned work and a description of any work to be undertaken in a future project.
\item Project Risks and their Mitigation; a detailed risk assessment.
\end{itemize}
\pagebreak

\end{abstract}

\tableofcontents
\listoffigures
\listofcapequs
\listoftables

\end{frontmatter}

% Increase default spaces between paragraphs
% \setlength{\parindent}{4em}
% \setlength{\parskip}{1em}

\chapter{System Overview}
\section{Introduction}
Tiberius finished its third major development cycle, in which there has been significant mechanical and software changes. The chapter will give a brief overview of the system, and the subsequent chapters will describe in depth the seven main components that were focused on. 

The seven main chapters are briefly described in figure (TODO: use figure from last report but change it up). Additionally there is another chapter chapter assigned to miscellaneous work.

\section{Software Overview}

Our software is primarily written in Python, with a small amount of C++ running on various microcontrollers.
All software is under version control on Github (\url{https://github.com/IonSystems/tiberius-robot}), and included version history as far back as 22nd January 2015 \cite{oldest-commit}.

\subsection{Software Modules}
Our software is split into many modules, each of which is briefly described below:
\begin{description}[align=left]
\item [Control] Contains drivers and control functions to communicate with all hardware devices. The module is structured with three layers of abstraction:
\begin{description}
\item [Integration-Level] Defines control loops and logic encompassing multiple specialised devices that were defined at the previous level.

\item [Specialised-Level] Defines the functionality of each hardware device, makes use of the drivers defined at the Driver-Level to communicate with the devices. Functions defined at this level only have access to the specialised driver for a particular device. 

\item [Driver-Level] Interacts with hardware at the most primative level. Does not contain any complex calculations or sophisticated behaviour, and data is not modified in any way to cause loss in precision.
\end{description}

\item [Web Interface] contains the Django Web Interface codebase. Does not depend on any other Tiberius modules.

\item [Database] provides functions to interact with the database.

\item [Database Wrapper] is a general purpose database wrapper, bridging the gap between the database engine and easy to write functions. This module can be used in any context (not just on Tiberius) and is designed to be extensible, so other database engines can be supported in future versions.

\item [Logging] provides a consistent logging behavior throughout all Tiberius software. All logging information is stored to the same file, with the source of the message clearly stated.

\item [Utilities] Contains miscellaneous functions that may come in handy in any other module.

\item [Configuration] allows the configuration file to be read from any other Tiberius module. 

\item [Navigation] contains an A-Start algorithm implementation, as well as basic navigation algorithms, allowing Tiberius to navigate using the GPS.

\item [Testing] is a central testing module, containing unit tests for all other Tiberius modules. Also contains useful developer scripts designed to initialise particular classes and print out logging information.

\item [Validation] contains validation logic to determine the status of database sensor information.

\item [Diagnostics] an incomplete module designed to contain a diagnosis thread. This thread is designed to run in the background, constantly monitoring sensor data and looking for pre-defined patterns. If a pattern was matched, then the module would diagnose a problem, and indicate the problem via status LED's and log messages. This module is further discussed in Future Work.

\item [Android Control]contains a 'legacy' android application developed in the previous year to ours. This application hasn't been touched, and not currently supported by the Control API. This module is also discussed further in the Future Work section.

\end{description}
\section{Hardware Overview}

\subsection{Sensors}
\subsubsection{Ultrasonics}
Active ultrasonic sensors generate high-frequency sound waves and evaluate the echo which is received back by the sensor, measuring the time interval between sending the signal and receiving the echo to determine the distance to an object.
\begin{figure}[!htb]
\begin{center}
\includegraphics[width=3cm]{ultrasonics.jpg}
\end{center}
\caption{Ultrasonics}
\label{fig:Ultrasonics}
\end{figure}

\subsubsection{Motor Encoders}
A rotary encoder, also called a shaft encoder, is an electro-mechanical device that converts the angular position or motion of a shaft or axle to an analog or digital code. 

\subsubsection{Kinect}
A powerful camera and microphone. The camera can provide a colour stream, a depth stream and an infra-red stream. The array microphone allows sound localisation. 
\begin{figure}[!htb]
\begin{center}
\includegraphics[width=3cm]{kinect.jpg}
\end{center}
\caption{kinect}
\label{fig:kinect}
\end{figure}

\subsection{Compute Power}
\subsubsection{Raspberry Pi}
The Raspberry Pi is a low cost, credit-card sized computer that plugs into a computer monitor, and uses a standard keyboard and mouse. It has 20 GPIO pins; these allow it to be connected to an I2C network. 
\begin{figure}[!htb]
\begin{center}
\includegraphics[width=3cm]{raspberrypi.jpg}
\end{center}
\caption{raspberrypi}
\label{fig:raspberrypi}
\end{figure}

\subsubsection{Kinect PC}
The Kinect PC is a motherboard with an attached SSD. It requires a USB 3.0 connection to interface with the Kinect. 

\subsubsection{SSD}
A storage device containing non-volatile flash memory, used in place of a hard disk because of its much greater speed. This is used in conjunction with a PC.

\subsubsection{mbed}
ARM® mbed™ IoT Device Platform, simply, is for writing software that controls hardware that can connect to the cloud - it is an easy way of creating embedded connected solutions.
\begin{figure}[!htb]
\begin{center}
\includegraphics[width=3cm]{mbed.jpg}
\end{center}
\caption{mbed}
\label{fig:mbed}
\end{figure}

\subsubsection{Arduino}
A pre-assembled Arduino board includes a microcontroller, which is programmed using Arduino programming language and the Arduino development environment. In essence, this platform provides a way to build and program electronic components. Arduino programming language is a simplified from of C/C++ programming language based on what Arduino calls "sketches," which use basic programming structures, variables and functions. These are then converted into a C++ program. 

Two versions of Arduino are on-board Tiberius. One is the nano and the other is a mega.

\begin{figure}[!htb]
\begin{center}
\includegraphics[width=3cm]{arduinonano.png}
\end{center}
\caption{arduino nano}
\label{fig:arduinoNano}
\end{figure}

\begin{figure}[!htb]
\begin{center}
\includegraphics[width=3cm]{ARDUINO_MEGA.png}
\end{center}
\caption{Arduino mega}
\label{fig:ArduinoMega}
\end{figure}

\subsubsection{RAMPS board}
Tiberius implements a 1.4 ramps board.
\begin{figure}[!htb]
\begin{center}
\includegraphics[width=3cm]{ramps.jpg}
\end{center}
\caption{ramps}
\label{fig:ramps}
\end{figure}

\subsubsection{MD03 Motor drivers}
% http://www.robot-electronics.co.uk/htm/md03tech.htm 
The MD03 is a medium power motor driver, designed to supply power beyond that of any of the low power single chip H-Bridges that exist. Main features are ease of use and flexibility. The motor's power is controlled by Pulse Width Modulation (PWM) of the H-Bridge at a frequency of 15khz ( 7.5khz before version 12). 
\begin{figure}[!htb]
\begin{center}
\includegraphics[width=3cm]{md03.jpg}
\end{center}
\caption{md03}
\label{fig:md03}
\end{figure}



\subsection{Actuators}
\subsubsection{Robot Arm}
The robotic arm has 3 joints and a free hanging gripper. It has a lifting capacity of 2Kg. It is controlled by an Arduino and can be either controlled using joint positions or Cartesian coordinates.
\subsubsection{Motors}
% http://www.mfacomodrills.com/gearboxes/986d_series.htm
% this link has the motor info on it.
Designed for heavy-duty industrial and model applications this robust unit boasts a powerful high quality motor with scintered bronze bearings. The metal gearbox incorporates ballrace bearings, enabling the high torque transfer from the motor to be
transmitted through the gearbox. The extended rear motor shaft can facilitate encoder installation \cite{Dun_dcmotor}.
\begin{figure}[!htb]
\begin{center}
\includegraphics[width=3cm]{motor.jpg}
\end{center}
\caption{motor}
\label{fig:motor}
\end{figure}



\pagestyle{blank}
\begin{appendices}
\chapter{Mechanical Documents}
\label{App:mech}
\pagestyle{duncan}

\subsection{Wishbone}
\includepdf[scale=0.8]{appendices/documents/Components_1.pdf}[!htb]
\subsection{Wheel knuckle}
\includepdf[scale=0.8]{appendices/documents/Components_2.pdf}
\subsection{Wheel component assembly}
\includepdf[scale=0.9]{appendices/documents/WheelsPartList.pdf}
\chapter{Comparison of IMDB Implementations}
\pagestyle{aidan}

\section{Test Results}
\subsection{Create}
Each database must be created before any operations can be carried out on it. The creation if a one off event that occurs at the start, therefore the time delay does not have significant importance.  

\begin{table}[!htb]
\centering

\label{Create}
\begin{tabular}{lllll}
     & RP1        & RP2 & RP3 &  \\
Poly & 0.00338196 & N/A    & 0.00092792    &  \\
SQL  & 0.03206205 & N/A    & 0.04080892    &  \\
     &             &     &     & 
\end{tabular}
\caption{Create database (seconds) }
\end{table}

\subsection{Insert}
New data received from the sensors must be inserted into the database to be stored. This action is done regularly and there the time delay is important. 
\begin{table}[!htb]
\centering

\label{insert}
\begin{tabular}{lllll}
     & RP1         & RP2 & RP3 &  \\
Poly & 0.00236365 & N/A    & 0.0002626    &  \\
SQL  & 0.03206205 & N/A    & 0.0577037    &  \\
     &             &     &     & 
\end{tabular}
\caption{Insert into database (seconds)}
\end{table}

\subsection{Update}
On certain occasions data in the table may need to be updated. This differs from inserted information as a new row of information is not appended however an existing row is searched for and it values are changed. This action is not done as often as inserting or querying although more often than creating and deleting.
\begin{table}[!htb]
\centering

\label{Update}
\begin{tabular}{lllll}
     & RP1         & RP2 & RP3 &  \\
Poly & 0.00347629 & N/A    & 0.00042764    &  \\
SQLite  & 0.03939703 & N/A    & 0.04500420    &  \\
     &             &     &     & 
\end{tabular}
\caption{Update the database (seconds)}
\end{table}

\subsection{Query}
To retrieve the information stored in the database it has to be queried (searched). This is done regularly and the time delay is important.
\begin{table}[!htb]
\centering

\label{Query}
\begin{tabular}{lllll}
     & RP1         & RP2 & RP3 &  \\
Poly & 0.138572986 & N/A    & 0.02222729 &  \\
SQLite  & 0.051900138 & N/A    & 0.00766252 &  \\
     &             &     &     & 
\end{tabular}
\caption{Query into database (seconds)}
\end{table}

\subsection{Delete}
The database may need to be deleted under certain circumstances. This will not happen often and therefore the delay is not important.
\begin{table}[!htb]
\centering

\label{Delete}
\begin{tabular}{lllll}
     & RP1         & RP2 & RP3 &  \\
Poly & 0.001752495 & N/A    & 0.00370311    &  \\
SQLite  & 0.068341016 & N/A    & 0.07257699    &  \\
     &             &     &     & 
\end{tabular}
\caption{Delete the database (seconds)}
\end{table}


\iffalse
The chosen RPi we are using is ...
For this RPi this bar graph which seperate bar for each operation and shaded for SQLite and unshaded for poly.
\fi
\input{appendices/app_mission_models.tex}
\chapter{Web Interface Request Code Walkthrough}
\label{app:web_walkthrough}
\pagestyle{cameron}

This walkthrough will provide a detailed description of all the steps that take place to communicate a command from the web interface to Tiberius. A standard configuration is assumed. A robotic arm command will be used as the example throughout this walkthrough.
\newline

\begin{enumerate}

\item The sequence of events is initiated by the user pressing a button on the web interface. In order to produce a button on the web page, the following \gls{HTML} code is used. See Code Listing \ref{cl:html_button}.
\newline

\begin{lstlisting}[caption=HTML Button, label=cl:html_button]
<button id="arm_button_y_minus" class="btn btn-white" title="Y -">
    <span class="fa fa-2x fa-arrow-down"></span>
</button>
\end{lstlisting}

\item The below jQuery function is executed when the button is clicked. The send\_command function, with specific parameters for the button clicked. See Code Listing \ref{cl:html_listener}.
\newline

\begin{lstlisting}[language=JavaScript, caption=jQuery Click Listener, label=cl:button_listener]
$('#arm_button_y_minus').click(function() {
		send_command(ip_address, 'arm_dy', -d);
});
\end{lstlisting}

\item The send\_command function sends a \gls{HTTP} \gls{POST} request to the \textit{web server}, passing on the parameters into the request. Note that we cannot directly message Tiberius's API at this point, as we are client-side. See Code Listing \ref{cl:js_ajax_post}.
\newline

\begin{lstlisting}[language=JavaScript, caption=Javascript message to web server, label=cl:js_ajax_post]
function send_command(ip_address, command_name, command_value){
	$.ajax({
			url: '../send_arm_request',
			type: 'POST',
			data: {
				'command_name':command_name,
				'command_value':command_value,
				'ip_address':ip_address
			},
			success: function (result) {
				//alert("anything");
			},
			error: function(error_msg){
				alert(error_msg);
			}
	});
}
\end{lstlisting}

\item The Django web app forwards the \gls{POST} request (with a matching URL) to the send\_arm\_request view. See Code Listing \ref{cl:django_urls}.
\newline

%from web interface control urls.py
\begin{lstlisting}[style=custompython, caption=Django URL Dispatcher, label=cl:django_urls]
urlpatterns = [
  url(r'^$', views.index, name='index'),
  url(r'^(?P<id>\d+)/', views.control, name='control'),
  url(r'^send_control_request', views.send_control_request,
  name='send_control_request'),
  url(r'^send_task_request', views.send_task_request,
  name='send_task_request'),
  url(r'^send_arm_request', views.send_arm_request,
  name='send_arm_request'),
]
\end{lstlisting}

\item The send\_arm\_request view forwards the request over the \gls{VPN}. Because Javascript code is executed client-side, we have no guarantee that the particular client will be connected to the VPN, so all requests are relayed through the web server to guarantee access to the \gls{VPN}.

Some security measures are in place here, the request to the web server must be a \gls{POST} request and a valid authentication token must be present. See Code Listing \ref{cl:django_arm_view}.
\newline

  % From web interface control views file.
  \begin{lstlisting}[style=custompython, caption=Django Arm Request Forwarding View, label=cl:django_arm_view]
  @require_http_methods(["POST"])
  def send_arm_request(request):
      headers = {'X-Auth-Token': settings.SUPER_SECRET_PASSWORD}

      # Contruct url for motor resource on Control API
      ip_address = request.POST.get('ip_address')
      url_start = "http://"
      url_end = ":8000/arm"
      url = url_start + ip_address + url_end
      response = ""

      try:
          r = requests.post(url,
                            data=request.POST.lists(),
                            headers=headers)
          response = r.text
      except ConnectionError as e:
          response = e
      return HttpResponse(response)
  \end{lstlisting}

\end{enumerate}


\input{appendices/app_authentication_middleware.tex}
\chapter{API Request Handling Code}
\label{app:on_post_code}

\begin{lstlisting}[style=custompython]
class MotorResource(object):

    def __init__(self, motor_control):
        self.motor_control = motor_control
        self.logger = logging.getLogger('tiberius.control_api.MotorResource')
        self.speed = 50
        self.state = MotorStates.STOP

    @falcon.after(generate_response)
    @falcon.before(validate_params)
    def on_post(self, req, resp):
	# Debug
	self.logger.debug("Request Params: " + str(req.params))

        # Can't go forwards and backwards at the same time so we can use elif.
        if(MotorStates.FORWARD in req.params):
            self.proc_forward()
        elif(MotorStates.BACKWARD in req.params):
            self.proc_backward()

        # Can't go left and right at the same time so elif.
        if(MotorStates.LEFT in req.params):
            self.proc_left()
        elif(MotorStates.RIGHT in req.params):
            self.proc_right()

        # Change the set speed of the motors.
        if('speed' in req.params):
            self.proc_speed(req.params['speed'])

        # Keep STOP at the bottom so nothing can overwrite it!
        if(MotorStates.STOP in req.params):
            self.proc_stop()

        # if(Commands.GET_SPEED in req.params):
        #     self.ret_speed()

    def proc_forward(self):
        self.motor_control.setSpeedPercent(self.speed)
        self.motor_control.moveForward()
        self.state = MotorStates.FORWARD
        self.logger.debug("Moving forward at speed %s", self.speed)

    def proc_backward(self):
        # speed = int(req.params['backward'])
        self.motor_control.setSpeedPercent(self.speed)
        self.motor_control.moveBackward()
        self.state = MotorStates.BACKWARD
        self.logger.debug("Moving backward at speed %s", self.speed)

    def proc_left(self):
        self.motor_control.setSpeedPercent(self.speed)
        self.motor_control.turnLeft()
        self.state = MotorStates.LEFT
        self.logger.debug("Turning left at speed %s", self.speed)

    def proc_right(self):
        self.motor_control.setSpeedPercent(self.speed)
        self.motor_control.turnRight()
        self.state = MotorStates.RIGHT
        self.logger.debug("Turning right at speed %s", self.speed)

    def proc_stop(self):
        self.motor_control.stop()
        self.state = MotorStates.STOP
        self.logger.debug("Stopped")

    def proc_speed(self, speed):
        self.speed = int(speed)
        self.logger.debug("Setting speed to %s", self.speed)
        # Now that the speed has been updated,
        # reinitiate any active states.
        if self.state == MotorStates.FORWARD:
            self.proc_forward()
        elif self.state == MotorStates.BACKWARD:
            self.proc_backward()
        elif self.state == MotorStates.LEFT:
            self.proc_left()
        elif self.state == MotorStates.RIGHT:
            self.proc_right()

    def ret_speed(self):
        return self.speed

\end{lstlisting}

\input{appendices/app_algorithms_py.tex}
\input{appendices/app_a_star_algorithm.tex}
\chapter{Web Interface Screenshots}
\label{app:web_screenshots}
\pagestyle{cameron}

\begin{figure}[!htb]
\begin{center}
\includegraphics[width=10cm]{web_arm_control.png}
\end{center}
\caption{Web Interface: Arm Controls}
\label{fig:web_arm_control}
\end{figure}

\begin{figure}[!htb]
\begin{center}
\includegraphics[width=10cm]{web_motor_control.png}
\end{center}
\caption{Web Interface: Motor Controls}
\label{fig:web_motor_control}
\end{figure}


\begin{figure}[!htb]
\begin{center}
\includegraphics[width=10cm]{web_mission_tasks.png}
\end{center}
\caption{Web Interface: Mission Tasks}
\label{fig:web_mission_tasks}
\end{figure}

\begin{figure}[!htb]
\begin{center}
\includegraphics[width=10cm]{web_plotting.png}
\end{center}
\caption{Web Interface: Plotting Tool}
\label{fig:web_plotting}
\end{figure}

\begin{figure}[!htb]
\begin{center}
\includegraphics[width=10cm]{web_waypoint_info.png}
\end{center}
\caption{Web Interface: Plotting Tool Waypoint Info}
\label{fig:web_waypoint_info}
\end{figure}

\begin{figure}[!htb]
\begin{center}
\includegraphics[width=10cm]{web_webcam.png}
\end{center}
\caption{Web Interface: Webcam Live Feed}
\label{fig:web_webcam}
\end{figure}

\begin{figure}[!htb]
\begin{center}
\includegraphics[width=10cm]{web_fleet.png}
\end{center}
\caption{Web Interface: Fleet View}
\label{fig:web_fleet}
\end{figure}
\chapter{User Guide}
\label{app:user_guide}
\includepdf[pages={-}]{appendices/documents/user_guide.pdf}
\end{appendices}

\backmatter
\pagestyle{blank}			%Set the page style to blank for the rest of the document.
\bibliography{biblio}{}		%Define bibliography file
\bibliographystyle{unsrt}

\printglossaries
\end{document}